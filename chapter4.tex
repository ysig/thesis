\chapter{Πειραματική Αξιολόγηση}
\label{chap4}
Στο κεφάλαιο αυτό, θα παρουσιάσουμε μία πειραματική αξιολόγηση του λογισμικού \en{grakel}.
Στο πρώτο μέρος θα περιγράψουμε αναλυτικά την πειραματική διάταξη και την μετρική μέσω της οποίας θα αξιολογήσουμε συγκριτικά την απόδοση κάθε πυρήνα.
Στη συνέχεια θα παρουσιάσουμε τα σύνολα δεδομένων στα οποία θα εκτελέσουμε τα πειράματα, χωρίζοντας τα σε κατηγορίες με βάση το είδος της πληροφορίας των γράφων που παρέχονται.
Έπειτα θα παρουσιάσουμε τα αποτελέσματα των πειραμάτων, μαζί με μία σύντομη ``εμπειρική'' αξιολόγηση.
Στο τέλος, θα παρουσιάσουμε κάποια θεωρητικά και πρακτικά συμπεράσματα, που προκύπτουν από την αξιολόγηση του λογισμικού.
\section{Πειραματική Διάταξη}
\label{sec:es}
\selectlanguage{greek}
Για την αξιολόγηση του \en{grakel} εκτελέσαμε σε μεμονωμένους πυρήνες ενός \en{cluster}\footnote{Θα ήθελα να ευχαριστήσω θερμά τον καθηγητή Απόστολο Παπαδόπουλο, για την προθυμία, την υποστήριξη και την παραχώρηση πρόσβασης και χρήσης στο μηχάνημα \en{\texttt{hyperion}} του Αριστοτελείου Πανεπιστημίου Θεσσαλονίκης, για την εκτέλεση των πειραμάτων του \en{grakel}.} αποτελούμενου από 80 ``\en{$\text{Intel}^{\text{\textregistered}}$ $\text{Xeon}^{\text{\textregistered}}$ CPU E7 - 4860 @ 2.27GHz}'' και 8 \en{board} σύγχρονης μνήμης \en{DDR3 - 1067 MHz} συνολικού μεγέθους ``\en{1TB}'', τον υπολογισμό του πλήρους \en{kernel} πίνακα μέσω της μεθόδου \en{fit\_transform} σε ένα εύρος τιμών (βλέπε πίνακα \ref{tab:kernel_parametrization}) και μία σειρά από συνόλων δεδομένων.
Ένα όριο τοποθετήθηκε στο μέγιστο χρόνο εκτέλεσης κάθε υπολογισμού καθώς και στην μέγιστη μνήμη \en{RAM} που μπορούσε να χρησιμοποιηθεί. 
Συγκεκριμένα για όλους τους υπολογισμούς τοποθετήθηκε το όριο της μίας μέρας (συμβολίζεται ως \en{\texttt{TIMEOUT}}) και των $64$\en{\texttt{GB}} (συμβολίζεται ως \en{\texttt{OUT-OF-MEMORY}}), μαζί με την φόρτωση της εισόδου.
Ταυτόχρονα για την έγκυρη σύγκριση των πυρήνων ο μέγιστος αριθμός από \tl{threads} που χρησιμοποίησε η βιβλιοθήκη \tl{BLASS} ορίστηκε ίσος με $1$.
Σε όλους τους αλγορίθμους για τους οποίους η επαύξηση μίας παραμέτρου αύξανε ή κρατούσε σταθερή την πολυπλοκότητα μνήμης και υπολογισμού, μία τιμή προς δοκιμή αγνοήθηκε στην περίπτωση που για την προηγούμενη υπήρξε \en{\texttt{TIMEOUT}} ή \en{\texttt{OUT-OF-MEMORY}}.
Στη συνέχεια έχοντας κρατήσει τις επισημειώσεις κάθε στοιχείου της μήτρας πυρήνα επιχειρήσαμε \en{10-fold cross validation} σε ένα ταξινομητή \en{SVM} με βάση την μετρική της ευστοχίας (βλ. \ref{subsection:accuracy}).
Συγκεκριμένα χρησιμοποιήσαμε τον ταξινομητή \en{\texttt{sklearn.svm.SVC}} που μας δίνει την δυνατότητα να λύσουμε το πρόβλημα \en{SVM} παρέχοντας μία προϋπολογισμένη μήτρα \en{Gram}.
Κάθε \en{fold}, αφορά τον χωρισμό των δεδομένων σε δύο μέρη 90\% και 10\%, γνωστά ως \en{train set} και \en{test set}.
Τα \en{folds} ήταν κοινά για όλους τους πυρήνες που εκτελέστηκαν σε αυτό το \en{dataset}.
Για καθένα από τα 10 \en{fold}, χωρίζουμε το δείγμα εκπαίδευσης σε δύο μέρη με μεγέθη 90\%/10\% , γνωστά ως (\en{validation train/test set}) και στα οποία για ένα εύρος τιμών $C$, συγκεκριμένα το $\{10^{-7}, 10^{-5}, .., 10^{5}, 10^{7}\}$ (για το $C$ βλέπε εξίσωση \ref{eq:svm_basique}) εκπαιδεύουμε το \en{SVM} στο \en{validation train set}.
Για τον συνδυασμό παραμετροποίησης και $C$ που μεγιστοποιεί την μετρική ευστοχίας στο \en{validation test set} αποθηκεύουμε την τιμή της μετρικής ευστοχίας που προκύπτει από την εκπαίδευση του \en{SVM} στο \en{train-set} κατά την πρόβλεψη των τιμών στο \en{test-set}.
Υπολογίζουμε σαν συνολική τιμή της μετρικής ευστοχίας την μέση τιμή τους για όλα τα \en{fold} όπως προκύπτει κατ' αυτό τον τρόπο, υπολογίζοντας συνολικά την μέση τιμή και την διακύμανση αυτών των τιμών, για $10$ επαναλήψεις.
Τέλος καταγράφουμε με τον ίδιο τρόπο σε συγκριτικούς πίνακες για κάθε \en{dataset} την μνήμη, τον χρόνο, ως μέση τιμή και διακύμανση από τις μέσες τιμές όλων της μνήμης και του χρόνου εκτέλεση για τους πυρήνες εκείνων των παραμετροποιήσεων που μεγιστοποιούν την τιμή της μετρικής ευστοχίας στο \en{validation test set} καθενός \en{fold}.

\subsection{Μετρική Ευστοχίας}
\label{subsection:accuracy}
Για την αξιολόγηση των πυρήνων αναφέραμε πως χρησιμοποιούν την μετρική της ευστοχίας.
Συγκεκριμένα για ένα πρόβλημα δυαδικής ταξινόμησης με θετικά και αρνητικά δείγματα υπάρχουν τέσσερεις δυνατές προβλέψεις:
\begin{enumerate}
    \item \en{True Positive (TP)} - το σύστημα προβλέπει σωστά μία θετική κλάση για ένα παράδειγμα που είναι θετικό
    \item \en{True Negative (TN)} - το σύστημα προβλέπει σωστά μία αρνητική κλάση για ένα αρνητικό παράδειγμα
    \item \en{False Positive (FP)} - το σύστημα προβλέπει σωστά μία λανθασμένη κλάση για ένα λανθασμένο παράδειγμα
    \item \en{False Negative (FN)} - το σύστημα προβλέπει λανθασμένα μία θετική κλάση για ένα αρνητικό παράδειγμα
\end{enumerate}
Αυτή η πληροφορία συνήθως παρουσιάζεται σε ένα $2 \times 2$ πίνακα \textit{σύγχυσης} (\en{confusion matrix}), όπως απεικονίζεται στον πίνακα \ref{tab:conf_mat}.
Στη βάση των τεσσάρων παραπάνω προβλέψεων προκύπτουν διάφορες πολύ γνωστές μετρικές αξιολόγησης.
Αυτές οι μετρικές μετρούν ποσοτικά την επίδοση ταξινόμησης για μία μόνο μέθοδο σε ένα και μόνο σύνολο δεδομένων.
Στην ταξινόμηση γράφων, η πιο γνωστή μετρική είναι αυτή της ευστοχίας (\en{accuracy}), που ορίζεται ως:
\begin{equation}
    acc = \frac{TP + TN}{TP + TN + FP + FN}    
\end{equation}
Η ευστοχία υπολογίζει μία μέθοδο βάσει του τμήματος των προβλέψεων τις που είναι σωστές.
Το κύριο μειονέκτημα της ευστοχίας είναι ότι στην περίπτωση μη ισορροπημένων κατανομών κατηγορίας, μπορεί να πάρει τεχνητά υψηλές τιμές.
Για παράδειγμα, αν σε ένα πρόβλημα δυαδικής ταξινόμησης το $99$\% των παραδειγμάτων είναι θετικά, τότε ένας αλγόριθμος μπορεί να πετύχει $99$\% ευστοχία προβλέποντας μόνο την θετική κατηγορία!
Για την επίλυση αυτού του προβλήματος, έχουν προταθεί άλλες μετρικές αξιολόγησης.

\begin{table}[]
\centering
\en{\begin{tabular}{ll|l|l|}
\cline{3-4}
                                                                                              &          & \multicolumn{2}{l|}{Predicted (Class)} \\ \cline{3-4} 
                                                                                              &          & Positive          & Negative         \\ \hline
\multicolumn{1}{|l|}{\multirow{2}{*}{\begin{tabular}[c]{@{}l@{}}\rotatebox{90}{Actual}\end{tabular}}} & Positive & TP                & FN               \\ \cline{2-4} 
\multicolumn{1}{|l|}{}                                                                        & Negative & FP                & TN               \\ \hline
\end{tabular}}
\caption{Πίνακας σύγχυσης για ένα πρόβλημα δυαδικής ταξινόμησης.}
\label{tab:conf_mat}
\end{table}
Από την άλλη, όπως φαίνεται στον πίνακα \ref{tab:kernel_graph_characteristics}, η πλειοψηφία των συνόλων δεδομένων που χρησιμοποιείται στην ταξινόμηση γράφων είναι σύνολα δεδομένων δυαδικής ταξινόμησης και στις περισσότερες περιπτώσεις, οι κλάσεις είναι ισορροπημένες.
Λόγω αυτής της παρατήρησης και προκειμένου τα αποτελέσματα να είναι συγκρίσιμα με προηγούμενες μελέτες, χρησιμοποιήσαμε την ευστοχία ως μέτρο αξιολόγησης.

\subsection{Παραμετροποίηση Πυρήνων}
Στο σύνολο τους, οι πυρήνες μπορούν να διαχωριστούν με βάση το είδος γράφων που δέχονται, όπως φαίνεται στον πίνακα \ref{tab:kernel_graph_characteristics}.
Συνεπώς για κάθε κατηγορία συνόλου δεδομένων εκτελέσαμε ένα διαφορετικό σύνολο πυρήνων.
Συγκεκριμένα, όλοι οι πυρήνες μπορούν να εκτελεστούν σε σύνολα δεδομένων χωρίς επισημειώσεις και όλοι οι πυρήνες που δέχονται συνεχείς επισημειώσεις μπορούν να εκτελεστούν σε σύνολα δεδομένα διακριτών (βλέπε υποενότητες \ref{ssec:lab}, \ref{ssec:atr} αντίστοιχα).
Προκειμένου να υπολογιστεί η βέλτιστη ευστοχία, με την μέθοδο που παρουσιάστηκε στην πειραματική διάταξη, υπολογίσαμε για κάθε πυρήνα και σύνολα δεδομένων τα αποτελέσματα τους σε ένα εύρος παραμέτρων \ref{tab:kernel_parametrization}.
Οι τιμές των παραμέτρων επιλέχθηκαν, τόσο βάσει των τιμών που αναφέρονται στην βιβλιογραφία κατά την πειραματική τους αξιολόγηση, όσο και μέσω εμπειρικών κανόνων και εκτιμήσεων που έχουν προκύψει από την χρήση τους.

\newpage
\begin{table}[hbtp!]
\begin{adjustbox}{angle=0}
\resizebox{1.0\textwidth}{!}{
\begin{tabular}{|l|c|c|c|c|c|c|}
\hline
\multirow{2}{*}{Πυρήνες}                                                           & \multirow{2}{*}{Υποενότητα} & \multirow{2}{*}{Αναγνωριστικό} & \multicolumn{2}{l|}{\begin{tabular}[c]{@{}l@{}}Διακριτές\\ Επισημειώσεις\end{tabular}} & \multicolumn{2}{l|}{\begin{tabular}[c]{@{}l@{}}Συνεχείς\\ Επισημειώσεις\end{tabular}} \\ \cline{4-7} 
                                                                                   &                             &                                & Κόμβοι                                     & Ακμές                                     & Κόμβοι                                    & Ακμές                                     \\ \hline
Τυχαίων Περιπάτων                                                                  & \ref{ssec:rw}                 & \en{RW}                           & \checkmark*                                 & -                                         & -                                         & -                                         \\ \hline
Κοντινότερων Μονοπατιών                                                            & \ref{ssec:sp}                  & \en{SP}                           & \checkmark*                                 & -                                         & \checkmark*                                & -                                         \\ \hline
Γραφιδίων                                                                          & \ref{ssec:gr}                  & \en{GR}                           & -                                          & -                                         & -                                         & -                                         \\ \hline
\begin{tabular}[c]{@{}l@{}}Πολυκλιμακωτός\\ Λαπλασιανός\end{tabular}               & \ref{ssec:ml}                  & \en{ML}                           & -                                          & -                                         & \checkmark                                 & -                                         \\ \hline
\begin{tabular}[c]{@{}l@{}}Ταιριάσματος\\ Υπογράφων\end{tabular}                   & \ref{ssec:sm}                  & \en{SM}                           & \checkmark*                                 & \checkmark*                                & \checkmark*                                & \checkmark*                                \\ \hline
\en{Lovasz $\vartheta$}                                                               & \ref{ssec:lovasz}              & \en{$\text{L}_{\vartheta}$}           & -                                          & -                                         & -                                         & -                                         \\ \hline
\en{SVM $\vartheta$}                                                                  & \ref{ssec:svm_k}                 & \en{$\text{SVM}_{\vartheta}$}         & -                                          & -                                         & -                                         & -                                         \\ \hline
\begin{tabular}[c]{@{}l@{}}Κατακερματισμού\\ Γειτονιάς\end{tabular}                & \ref{ssec:nh}                  & \en{NH}                           & \checkmark                                  & -                                         & -                                         & -                                         \\ \hline
\begin{tabular}[c]{@{}l@{}}Αποστάσεων ζευγαριών\\ υπογράφων γειτονιάς\end{tabular} & \ref{ssec:nspdk}               & \en{NSPDK}                        & \checkmark                                  & -                                         & -                                         & -                                         \\ \hline
\en{ODD-STh}                                                                         & \ref{ssec:odd-sth}            & \en{ODD-STh}                      & \checkmark                                  & -                                         & -                                         & -                                         \\ \hline
Διάδοσης                                                                           & \ref{ssec:p2k}                         & \en{P2K}                          & \checkmark                                  & -                                         & \checkmark                                 & -                                         \\ \hline
\begin{tabular}[c]{@{}l@{}}Πυραμιδικού\\ ταιριάσματος\end{tabular}                 & \ref{ssec:pm}                         & \en{PM}                           & \checkmark*                                 & -                                         & -                                         & -                                         \\ \hline
\begin{tabular}[c]{@{}l@{}}Ταιριάσματος\\ Ιστογραμμάτων\end{tabular}               & \ref{ssec:wl}                         & \en{VH}                           & \checkmark                                  & -                                         & -                                         & -                                         \\ \hline
Αλμάτων Γράφων                           & \ref{ssec:gh}                         & \en{GH}                           & -                                          & -                                         & \checkmark                                 & -                                         \\ \hline
Σκελετός \en{Weisfeiler-Lehman}                                                   & \ref{ssec:wl}                         & \en{WL}                           & \checkmark                                  & -                                         & -                                         & -                                         \\ \hline
Σκελετός \en{Core}                                                                & \ref{ssec:core}                     & \en{CORE}                     & -                                          & -                                         & -                                         & -                                         \\ \hline
\end{tabular}
}
\end{adjustbox}
\caption[Χωρισμός των πυρήνων γράφων που χρησιμοποιήθηκαν για τα πειράματα με βάση το είδος των επισημειώσεων που περιμένουν στις εισόδους τους.]{Χωρισμός των πυρήνων γράφων που χρησιμοποιήθηκαν για τα πειράματα με βάση το είδος των επισημειώσεων που περιμένουν στις εισόδους τους. Με \checkmark συμβολίζουμε την περίπτωση όπου ένα πυρήνας περιμένει οι γράφοι της εισόδου του να έχουν αυτού του τύπου την επισημείωση, ενώ \checkmark* όταν μπορεί να λειτουργήσει και χωρίς, υλοποιώντας έναν σχετικά διαφορετικό αλγόριθμο. Στην περίπτωση που ένας πυρήνας δέχεται γράφους εισόδου με \en{\checkmark} και στις συνεχείς και στις διακριτές επισημειώσεις αυτό αφορά δύο σχετικά διαφορετικούς αλγορίθμους. Ακόμα παρατίθενται τα αναγνωριστικά που χρησιμοποιούνται στους πίνακες των πειραμάτων καθώς και η υποενότητα στην οποία περιγράφεται καθένας από τους πυρήνες.}
\label{tab:kernel_graph_characteristics}
\end{table}

\begin{table}[htp]
\centering
\begin{adjustbox}{angle=0}
\resizebox{1.0\textwidth}{!}{
\begin{tabular}{|l|l|l|}
\hline
\multirow{2}{*}{Πυρήνες} & \multicolumn{2}{l|}{Παραμετροποίηση} \\ \cline{2-3} 
                         & Σταθερή           & Κινητή           \\ \hline
\en{RW}                       &$\lambda = 10^{\lceil \log_{10}(\frac{1}{\delta^{2}_{\max}}) \rceil} $                 &   $p \in \{2, \dots 10, \inf\}$              \\ \hline
\en{SP}                       &-                   &-                  \\ \hline
\en{GR}                       &$k=5$                   &$n_{\text{\en{samples}}}=\{200, 500, 1000, 2000, 5000\}$                  \\ \hline
\en{ML}                       &$\gamma=0.01, \eta=0.01, P=10$                &  $L\in\{0, \dots, 5\}, N\in\{50, 100, 200, 300\} $                \\ \hline
\en{SM}                       &$k=3$                   &    -              \\ \hline
\en{$\text{L}_{\vartheta}$}      &$2\leq|S|\leq 8$                &  $n_{\text{\en{samples}}}=\{100, 200, 500, 1000\}$             \\ \hline
\en{$\text{SVM}_{\vartheta}$}  &$2\leq|S|\leq 8$  &  $n_{\text{\en{samples}}}=\{100, 200, 500, 1000\}$                \\ \hline
\en{NH}                       &\en{CS-NH}             & $R\in\{1, \dots, 6\}$                 \\ \hline
\en{NSPDK}                    &-                   &  $r\in\{1, \dots, 6\}, d\in\{3, \dots, 7\}$               \\ \hline
\en{ODD-STh}                  &-                   & $h\in\{1, \dots, 11\}$                \\ \hline
\en{PK}                      &$w=10^{-5}$                   & $t_{\max}\in \{1, \dots, 6\}$                \\ \hline
\en{PM}                       &-                   &$L \in \{2, 4, 6\}, d \in \{4, 6, 8, 10\}$                  \\ \hline
\en{GH}                       &-                   &\en{linear/gaussian-kernel}                  \\ \hline
\en{VH}                       &   -                & -                 \\ \hline
\en{WL}                       &  -                 & $n_{\text{\en{iter}}} \in \{4, \dots, 8\} $                 \\ \hline
\en{CORE}                     &  -                 & -                \\ \hline
\end{tabular}
}
\end{adjustbox}
\caption[Οι παραμετροποιήσεις των πυρήνων, που επιλέχθηκαν για την πειραματική τους αξιολόγηση.]{Οι παραμετροποιήσεις των πυρήνων, που επιλέχθηκαν για την πειραματική τους αξιολόγηση. Η σταθερή παραμετροποίηση αφορά τιμές παραμέτρων του πυρήνα που ήταν σταθερές για όλη την εναλλαγή τιμών της κινητής. Οι τιμές της κινητής παραμετροποίησης συνεπάγονται έναν υπολογισμό της μήτρας πυρήνα για όλα τα δυνατά ζευγάρια παραμετροποιήσεων που προκύπτουν από τα ζευγάρια που σημειώνονται, ενώ σε περίπτωση απουσίας ``-'' η μήτρα πυρήνα υπολογίστηκε μία φορά. Με $\delta_{\max}$ συμβολίζουμε τον μέγιστο βαθμό του αντίστοιχου συνόλου δεδομένων στο οποίο υπολογίζεται η μήτρα πυρήνα.}
\label{tab:kernel_parametrization}
\end{table}
\newpage
\section{\en{Datasets}}
\label{sec:datasets}
Η παραπάνω πειραματική διάταξη εφαρμόστηκε σε ένα μεγάλο εύρος συνόλου δεδομένων και πυρήνων γράφων με βάση το είδος τους.
\subsection{Χωρίς Επισημειώσεις}
\label{ssec:unlabeled}
Για την αξιολόγηση των πυρήνων που δέχονται στην είσοδό τους γράφους χωρίς επισημειώσεις, χρησιμοποιήσαμε τα παρακάτω σύνολα δεδομένων.
Προκειμένου οι πυρήνες που είναι σχεδιασμένοι για γράφους με διακριτές και συνεχείς επισημειώσεις, να είναι εκτελέσιμοι στα ίδια σύνολα δεδομένων, αποδώσαμε σε κάθε κόμβο ή ακμή μία σταθερή επισημείωση (τον αριθμό $1$) και ένα μοναδιαίο διάνυσμα ενός στοιχείου (\en{\texttt{numpy.array([1.0])}}) αντίστοιχα.

\paragraph*{\en{COLLAB}} Μία συλλογή δεδομένων επιστημονικής συνεργασίας που αποτελείται από τα \textit{δίκτυα προσωπικότητας} (\en{ego-networks}) αρκετών ερευνητών από τρία υποπεδία της φυσικής (Φυσική Υψηλών Ενεργειών, Φυσική Στερεάς Κατάστασης και της Αστροφυσικής).
Ο σκοπός είναι είναι να προσδιοριστεί το υποπεδίο της φυσικής στο οποίο ανήκει το δίκτυο προσωπικότητας του κάθε ερευνητή \cite{DGK_PINAR}.

\paragraph*{\en{IMDB-BINARY, IMDB-MULTI}}
Αυτά τα σύνολα δεδομένων δημιουργήθηκαν από το \en{IMDb} (\en{\texttt{\href{www.imdb.com}{https://www.imdb.com}}}), μία \en{online} βάση δεδομένων με πληροφορίες που συνδέονται με ταινίες και προγράμματα τηλεόρασης.
Οι γράφοι που περιέχονται στα δύο σύνολα δεδομένων αντιστοιχούν σε συνεργασίες ηθοποιών σε ταινίες.
Έτσι, οι κόμβοι κάθε γράφου αναπαριστούν ηθοποιούς και δύο κόμβοι συνδέονται με μία ακμή αν οι αντίστοιχοι ηθοποιοί παίζουν στην ίδια ταινία.
Κάθε γράφος είναι ένα  δίκτυο προσωπικότητας ηθοποιών και ο στόχος είναι η πρόβλεψη της κατηγορίας ταινιών (\en{genre}) στην οποία ανήκει μία ταινία \cite{DGK_PINAR}.

\paragraph*{\en{REDDIT-BINARY, REDDIT-MULTI-5k, REDDIT-MULTI-12k}}
Οι γράφοι που περιέχονται σε αυτά τα τρία σύνολα δεδομένων αναπαριστούν κοινωνικές αλληλεπιδράσεις μεταξύ χρηστών του \en{Reddit} (\en{\texttt{\href{www.reddit.com}{https://www.reddit.com}}}), ένα από τα πιο δημοφιλή μέσα κοινωνικής δικτύωσης.
Κάθε γράφος αναπαριστά ένα νήμα συζήτησης στον ιστό.
Συγκεκριμένα, κάθε κόμβος αντιστοιχεί σε ένα χρήστη και δύο χρήστες συνδέονται από μία ακμή αν τουλάχιστον ένας από αυτούς αντέδρασε στο σχόλιο του άλλου.
Στόχος είναι η ταξινόμηση γράφων είτε σε κοινότητες είτε σε ``υπο-\en{reddit}'' (\en{subreddits}) \cite{DGK_PINAR}.

\subsection{Με Διακριτές Επισημειώσεις}
\label{ssec:lab}
Για την αξιολόγηση των πυρήνων που δέχονται στην είσοδο τους γράφους με \textbf{διακριτές} επισημειώσεις, χρησιμοποιήσαμε τα παρακάτω σύνολα δεδομένων.
Προκειμένου οι πυρήνες, που δέχονται στην είσοδο τους γράφους με συνεχείς επισημειώσεις, να είναι εκτελέσιμοι στα ίδια σύνολα δεδομένων, αποδώσαμε σε κάθε κόμβο ένα \en{\textit{One-Hot Vector}} με βάση το σύνολο όλων των επισημειώσεων που εμφανίζονται σε κάθε σύνολο δεδομένων.

\paragraph*{\en{MUTAG}} Αυτό το σύνολο δεδομένων αποτελείται από 188 μεταλλαξιογόνες αρωματικές-ετεροαρωματικές νιτρικές ενώσεις.
Ο στόχος είναι η πρόβλεψη του αν μία χημική ένωση έχει μεταλλαξιογόνα δράση στο αρνητικό κατά \en{Gram} βακτήριο \en{Salmonella typhimurium} \cite{shervashidze2011weisfeiler}.

\paragraph*{\en{ENZYMES}} Αποτελείται από $600$ τριτογενείς δομές πρωτεϊνών που ανήκουν στην βάση ενζύμων \en{BRENDA}.
Κάθε ένζυμο είναι ταξινομημένο στην αφηρημένη κατάταξη \en{\href{https://en.wikipedia.org/wiki/Enzyme_Commission_number}{Enzyme Commission}} και σκοπός είναι ο έγκυρος προσδιορισμός της κλάσης στην οποία ανήκει ένα ένζυμο \cite{Borgwardt2005}.

\paragraph*{\en{DD}} Αυτό το σύνολο δεδομένων περιέχει πάνω από χίλιες δομές πρωτεϊνών.
Κάθε πρωτεΐνη είναι ένας γράφος που οι κόμβοι του αντιστοιχούν σε αμινοξέα και ένα ζευγάρι αμινοξέων συνδέεται με μία ακμή αν η απόσταση τους είναι λιγότερη από 6 \en{Ångstrom}.
Στόχος είναι να προβλέψουμε αν μία πρωτεΐνη είναι ένζυμο (ή όχι) \cite{DobsonDoig03, shervashidze2011weisfeiler}.

\paragraph*{\en{NCI1}} Αυτό το σύνολο δεδομένων περιέχει μερικές χιλιάδες χημικά στοιχεία στα οποία καταγράφεται η δραστηριότητα τους απέναντι σε καρκινικά κύτταρα του πνεύμονα και των ωοθηκών βάση της πορείας κυτταρικής διαίρεσης τους (\en{cell lines}) σε ελεγχόμενες συνθήκες εργαστηρίου \cite{Wale2008}.

\paragraph*{\en{PTC\_MR}} Αυτό το σύνολο δεδομένων περιέχει $344$ οργανικά μόρια που έχουν αναπαρασταθεί ως γράφοι.
Στόχος είναι η πρόβλεψη καρκινογένεσης σε αρσενικούς αρουραίους \cite{Toivonen2003}.

\paragraph*{\en{AIDS}} Αποτελείται από $2000$ χημικές ενώσεις που έχουν αναπαρασταθεί ως γράφοι, οι οποίες έχουν δοκιμαστεί για την αποτελεσματικότητα τους απέναντι στον ιό \en{HIV}. Σκοπός λοιπόν του προβλήματος ταξινόμησης είναι η πρόβλεψη του κατά πόσον μία χημική ένωση μπορεί να είναι ή όχι αποτελεσματική απέναντι στον ιό \cite{Riesen08}.

\paragraph*{\en{PROTEINS}} Περιέχει πρωτεΐνες που έχουν αναπαρασταθεί ως γράφοι, όπου οι κόμβοι είναι δευτερογενή δομικά στοιχεία και μία ακμή υπάρχει μεταξύ των κόμβων αν οι κόμβοι είναι γείτονες στην ακολουθία αμινοξέων ή στον τρισδιάστατο χώρο.
Σκοπός είναι η ταξινόμηση μίας πρωτεΐνης ως ένζυμο (ή όχι) \cite{borgwardt2005protein}.

\subsection{Με Επισημειώσεις Χαρακτηριστικών}
\label{ssec:atr}
Για την αξιολόγηση πυρήνων που δέχονται στην είσοδο τους γράφους με \textbf{συνεχείς} επισημειώσεις, χρησιμοποιήσαμε τα παρακάτω σύνολα δεδομένων.

\paragraph*{\en{ENZYMES}} Αποτελείται από $600$ τριτογενείς δομές πρωτεϊνών που ανήκουν στην βάση ενζύμων \en{BRENDA}.
Κάθε ένζυμο είναι ταξινομημένο στην αφηρημένη κατάταξη \en{\href{https://en.wikipedia.org/wiki/Enzyme_Commission_number}{Enzyme Commission}} και ο σκοπός είναι ο ορθός προσδιορισμός της κλάσης στην οποία ανήκει ένα ένζυμο \cite{Borgwardt2005}.

\paragraph*{\en{Synthie}} Είναι ένα τεχνητό σύνολο δεδομένων που αποτελείται από $400$ γράφους.
Το σύνολο δεδομένων υποδιαιρείται σε $4$ κατηγορίες.
Κάθε κόμβος επισημειώνεται με ένα διάνυσμα $15$ στοιχείων.
Για την κατασκευή του παράγονται δύο σύνολα με διαφορετική παραμετροποίηση $200$ γράφων \en{Erdös-Rényi} όπου το 25\% των ακμών τους αφαιρείται τυχαία, ενώ κατηγοριοποιούνται σε δύο κλάσεις, διαλέγοντας και συνδέοντας τυχαία $10$ γράφους με πιθανότητα 0.8 και 0.2 από το πρώτο και το δεύτερο σύνολο αντίστοιχα για την πρώτη κατηγορία και με αντίστροφες πιθανότητες για την δεύτερη κατηγορία.
Έπειτα δημιουργώντας δύο σύνολα χαρακτηριστικών δεκαπενταδιάστατων διανυσμάτων δύο κατηγοριών, οι παραπάνω δύο κατηγορίες χωρίζονται σε άλλες δύο, όπου για την πρώτη υποκατηγορία, αν ένα κόμβος προερχόταν από το πρώτο σύνολο γράφων επισημειώνεται τυχαία με ένα διάνυσμα του πρώτου συνόλου χαρακτηριστικών ενώ σε αντίθετη περίπτωση με διάνυσμα του δεύτερου.
Για την παραγωγή της δεύτερης κατηγορίας συμβαίνει το αντίθετο.
Στόχος του προβλήματος ταξινόμησης είναι βάσει των χαρακτηριστικών, να ανιχνευθεί σε ποια από τις τέσσερεις υποκατηγορίες ανήκει ένας γράφος \cite{Morris16}.

\paragraph*{\en{BZR}} Αυτό το σύνολο δεδομένων αποτελεί από $684$ χημικές ενώσεις κατηγοριοποιημένες ως μεταλλαξιογόνες ή μη, βάσει ενός πειράματος γνωστό ως \en{Salmonella/microsome assay}.
Αυτό το σύνολο δεδομένων είναι ισοσταθμισμένο, με $341$ μεταλλαξιογόνες χημικές ενώσεις και $343$ μη-μεταλλαξιογόνες \cite{mahe2009graph, Neumann2016}.

\paragraph*{\en{PROTEINS\_full}} Αυτό το σύνολο δεδομένων αποτελείται $1113$ από χημικές ενώσεις προερχόμενες από την βάση δεδομένων πρωτεϊνών \en{PDB}.
Διαχωρισμένες σε ένζυμα ($59$\%) και μη-ένζυμα ($41$\%), οι πρωτεΐνες έχουν διαλεχτεί έτσι ώστε καμία ακολουθία να μην ταιριάζει με μία άλλη.
Παρέχουν πλούσια επισημείωση για κάθε κόμβο σε μορφή $29$-διάστατων χαρακτηριστικών χρησιμοποιώντας μεταξύ άλλων την κρυσταλλογραφική τους πληροφορία \cite{DobsonDoig03, borgwardt2005protein, Neumann2016}

\paragraph*{\en{SYNTHETICnew}} είναι ένα τεχνητό σύνολο δεδομένων $300$ τυχαία δειγματοληπτημένων γράφων που αποτελούνται από $100$ κόμβους και $196$ ακμές, στους κόμβους των οποίων ανατίθενται μονοδιάστατες συνεχείς επισημειώσεις από το $\mathcal{N}(0, 1)$.
Έπειτα δημιουργούνται δύο ισοσταθμισμένες κατηγορίες $150$ επισημειώσεων, αφαιρώντας και επαναπροσθέτοντας τυχαία $5$ ακμές και μεταθέτοντας τυχαία τις επισημειώσεις $10$ κόμβων για την πρώτη κατηγορία και $10$, $5$ για την δεύτερη προσθέτοντας στο τέλος τυχαίο θόρυβο σε όλες τις επισημειώσεις από την $\mathcal{N}(0, 0.452)$ \cite{Feragen13}.
\newline \\
Όλα τα σύνολα δεδομένων που αναφέρθηκαν παραπάνω προέρχονται από το \cite{KKMMN2016}.
Στατιστικά στοιχεία και πληροφορίες σχετικά με την ύπαρξη και τον τύπο των επισημείωσεων τους παρουσιάζονται συνοπτικά στον πίνακα \ref{ref:dataset_statistics}.

\begin{table}[]
\centering
\begin{adjustbox}{angle=90}
\resizebox{1\textheight}{!}{
\en{
\begin{tabularx}{0.853\textheight}{|l|c|c|c|c|c|c|}
\hline
\multirow{2}{*}{Dataset Name} & \multicolumn{4}{c|}{Statistics}                                                & \multicolumn{2}{c|}{Node-Labels/Node-Attributes (Dim.)}                            \\ \cline{2-7} 
      & \#Graphs & \#Classes & Avg. \#Nodes & Avg. \#Edges & Node-Lab. & Node-Attr.   \\ \hline
AIDS                          & 2000           & 2               & 15.69                & 16.20                & +           & + (4)                             \\ \hline
BZR                           & 405            & 2               & 35.75                & 38.36                & +           & + (3)                             \\ \hline
COLLAB                        & 5000           & 3               & 74.49                & 2457.78              & –           & –                                 \\ \hline
DD                            & 1178           & 2               & 284.32               & 715.66               & +           & –                                 \\ \hline
ENZYMES                       & 600            & 6               & 32.63                & 62.14                & +           & + (18)                            \\ \hline
IMDB-BINARY                   & 1000           & 2               & 19.77                & 96.53                & –           & –                                 \\ \hline
IMDB-MULTI                    & 1500           & 3               & 13.00                & 65.94                & –           & –                                 \\ \hline
MUTAG                         & 188            & 2               & 17.93                & 19.79                & +           & –                                 \\ \hline
PTC\_MR                       & 344            & 2               & 14.29                & 14.69                & +           & –                                 \\ \hline
PROTEINS                      & 1113           & 2               & 39.06                & 72.82                & +           & + (1)                             \\ \hline
PROTEINS\_full                & 1113           & 2               & 39.06                & 72.82                & +           & + (29)                            \\ \hline
REDDIT-BINARY                 & 2000           & 2               & 429.63               & 497.75               & –           & –                                 \\ \hline
REDDIT-MULTI-5k               & 4999           & 5               & 508.52               & 594.87               & –           & –                                 \\ \hline
REDDIT-MULTI-12k              & 11929          & 11              & 391.41               & 456.89               & –           & –                                 \\ \hline
SYNTHETICnew                  & 300            & 2               & 100.00               & 196.25               & –           & + (1)                             \\ \hline
Synthie                       & 400            & 4               & 95.00                & 172.93               & –           & + (15)                            \\ \hline
\end{tabularx}
}
}
\end{adjustbox}
\caption{Στατιστικά στοιχεία για τα σύνολα δεδομένων, όσον αφορά τους γράφους και τις επισημειώσεις.}
\label{ref:dataset_statistics}
\end{table}\newpage
\section{Αποτελέσματα \& Αξιολόγηση}
Σε αυτό το μέρος θα παρουσιάσουμε τα αποτελέσματα από όλα τα πειράματα όπως περιγράφηκαν παραπάνω.
Η αξιολόγηση θα αποτελείται από 3 πίνακες στους οποίους: ο πρώτος θα αποτελείται από τις επιδόσεις ευστοχίας των πυρήνων για όλους τους συνδυασμούς παραμέτρων όπως περιγράφονται στην ενότητα \ref{sec:es} και για όλες τις αντίστοιχες κατηγορίες από \en{dataset} όπως περιγράφονται στην ενότητα \ref{sec:datasets}, ενώ ο δεύτερος και ο τρίτος θα αποτελείται από τον χρόνο εκτέλεσης και την μέγιστη μνήμη για τις καταγραφόμενες στην ευστοχία εκτελέσεις.
Έτσι για κάθε σύνολο δεδομένων θα σημειώνονται οι αποτελεσματικότεροι πυρήνες, ενώ παράλληλα θα σχολιάζονται οι επιδόσεις τους σε σχέση με τις απαιτήσεις χρόνου και μνήμης των υπόλοιπων πυρήνων.
\subsection{Χωρίς Επισημειώσεις} 
Στα σύνολα δεδομένων \en{IMDB-BINARY, IMDB-MULTI, COLLAB, REDDIT-MULTI-12K, REDDIT-BINARY} και \en{REDDIT-MULTI-5K} με επισημειώσεις χαρακτηριστικών, όπως περιγράφονται στην υποενότητα \ref{ssec:unlabeled}, εκτελέστηκαν οι πυρήνες 
\en{GH, GR, $\text{L}_{\vartheta}$, ML, NH, NSPDK, ODD-STh, P2K, PM, RW, SM, SP, $\text{SVM}_{\vartheta}$} και \en{VH}.
Επιλέξαμε σε αυτό το σημείο να μην αξιολογήσουμε την χρήση κάποιου σκελετού πυρήνα, προκειμένου να ελέγξουμε πως  συμπεριφέρονται οι πυρήνες από μόνοι τους, σε πληροφορία που αντανακλά εξολοκλήρου τη δομή.
Καλύτερη επίδοση τόσο σε ευστοχία όσο και σε χρόνο και μνήμη, είχαν οι πυρήνες \en{NH}, \en{PM} και \en{ML} με τον πρώτο να παρουσιάζει μεγαλύτερη ευστοχία στα σύνολα δεδομένων \en{IMDB-BINARY, IMDB-MULTI, COLLAB} και \en{REDDIT-MULTI-5K}, \en{PROTEINS\_full}, τον δεύτερο στo \en{REDDIT-MULTI-12K} ενώ τον τελευταίο στο \en{REDDIT-BINARY} όπως φαίνεται στον πίνακα \ref{tab:acc:unlabelled}.
Ακόμα ο πυρήνας \en{SM} αδυνατεί να ανταποκριθεί σχεδόν σε όλα τα σύνολα δεδομένων τόσο λόγω χρόνου όσο και μνήμης, μιας και στην περίπτωση των γράφων χωρίς επισημειώσεις συνυπολογίζει όλες τις κλίκες.
Ο πυρήνας \en{$\text{L}_{\vartheta}$} ξεπερνάει το όριο χρόνου όπως είναι αναμενόμενο, καθώς τα σύνολα δεδομένων αυξάνονται, ενώ παράλληλα η προσέγγιση του από τον \en{$\text{S}_{\vartheta}$} φαίνεται να έχει καλύτερη επίδοση, χωρίς από την άλλη να πλησιάζει τις βέλτιστες τιμές.
Η μεγάλη αποτελεσματικότητα του πυρήνα \en{NH} αποτελεί μία ένδειξη της αποτελεσματικότητας, μίας επαναληπτικής διαδικασίας εμπλουτισμού των επισημειώσεων όπως αυτή του \en{Weisfeiler-Lehman} σε δεδομένα χωρίς επισημειώσεις.
Από την άλλη, ο υψηλός αριθμός ακμών φαίνεται να αποτυπώνεται στην μνήμη, λόγω του υψηλού αριθμού επισημειώσεων που προκύπτουν από την πολυπλοκότητα στην ίδια την δομή των γράφων στα αντίστοιχα σύνολα δεδομένων, χωρίς από την άλλη να προσαυξάνεται πάντοτε αναλογικά και ο χρόνος εκτέλεσης.
Οι αποτελεσματικότητα των πυρήνων \en{PM} και \en{ML} φαίνεται να οφείλεται στο ότι εξετάζουν την δομή πολυκλιμακωτά, που ενώ στην περίπτωση του πρώτου δεν αντανακλά σε ιδιαίτερα υψηλή μνήμη και χρόνο εκτέλεσης, κάτι τέτοιο φαίνεται να συμβαίνει στην περίπτωση του δεύτερου.
Το πείραμα σε μη επισημειωμένα δεδομένα είναι χρήσιμο για να παρατηρήσει κανείς την επίδοση των πυρήνων σε σύνολα δεδομένων μεγάλου μεγέθους, μιας και το μικρότερο απ' αυτά το \en{IMDB-BINARY} με $1000$ γράφους.
\begin{table}[]
\centering
\begin{adjustbox}{angle=90}
\resizebox{0.95\textheight}{!}{
\en{
\begin{tabular}{|l|c|c|c|c|c|c|}
\hline
\cellcolor[HTML]{CB0000} &  IMDB-BINARY &   IMDB-MULTI &    COLLAB & REDDIT-MULTI-12K & REDDIT-BINARY & REDDIT-MULTI-5K \\\hline
GH              &  57.69 ± 1.31 &  40.04 ± 0.91 &       60.21 ± 0.1 &          TIMEOUT &           TIMEOUT &           TIMEOUT \\\hline
GR              &  65.19 ± 0.97 &  39.82 ± 0.89 &      70.63 ± 0.25 &     23.06 ± 0.07 &      76.81 ± 0.34 &      34.26 ± 0.47 \\\hline
$\text{L}_{\vartheta}$   &  49.21 ± 1.33 &  39.33 ± 0.95 &           TIMEOUT &          TIMEOUT &           TIMEOUT &           TIMEOUT \\\hline
ML              &  70.94 ± 0.93 &  47.92 ± 0.87 &      75.29 ± 0.49 &          OUT-OF-MEMORY &       \cemph{89.3} ± 0.39 &       35.1 ± 0.29 \\\hline
NH              &  \cemph{73.34} ± 0.98 &   \cemph{50.68} ± 0.5 &      \cemph{79.99} ± 0.39 &      39.65 ± 0.2 &      81.53 ± 0.38 &      \cemph{49.34} ± 0.37 \\\hline
NSPDK           &  68.81 ± 0.71 &   45.1 ± 0.63 &  TIMEOUT &          TIMEOUT &           TIMEOUT &           TIMEOUT \\\hline
ODD-STh         &   64.7 ± 0.73 &   46.8 ± 0.51 &              52.0 &     29.83 ± 0.08 &      50.84 ± 0.97 &      42.99 ± 0.09 \\\hline
P2K             &  51.15 ± 1.67 &  33.15 ± 1.08 &      58.67 ± 0.15 &     24.07 ± 0.12 &      63.11 ± 0.94 &      34.54 ± 0.36 \\\hline
PM              &  66.67 ± 1.45 &  45.25 ± 0.79 &      74.57 ± 0.34 &     \cemph{41.15} ± 0.17 &      86.43 ± 0.51 &      48.34 ± 0.26 \\\hline
RW              &  63.87 ± 1.06 &  45.75 ± 1.03 &       68.0 ± 0.07 &  OUT-OF-MEMORY &  TIMEOUT &  TIMEOUT \\\hline
SM              &       TIMEOUT &       TIMEOUT &           TIMEOUT &          OUT-OF-MEMORY &           OUT-OF-MEMORY &           OUT-OF-MEMORY \\\hline
SP              &  55.18 ± 1.23 &  39.37 ± 0.84 &       58.8 ± 0.08 &          TIMEOUT &      81.67 ± 0.23 &      47.81 ± 0.23 \\\hline
$\text{SVM}_{\vartheta}$ &  51.35 ± 1.54 &    38.4 ± 0.6 &      55.72 ± 0.31 &     23.03 ± 0.18 &      74.71 ± 0.01 &      29.54 ± 0.51 \\\hline
VH              &   46.54 ± 0.8 &   29.59 ± 0.4 &              52.0 &            21.73 &      47.32 ± 0.66 &      17.92 ± 0.42 \\\hline
\end{tabular}
}
}
\end{adjustbox}
\caption[Μέσοι όροι και διακυμάνσεις της μετρικής ευστοχίας από $10$ επαναλήψεις  \en{$10$-fold cross validation} στα σύνολα δεδομένων χωρίς επισημειώσεις.]{\small Μέσοι όροι και διακυμάνσεις της μετρικής ευστοχίας από $10$ επαναλήψεις  \en{$10$-fold cross validation} στα σύνολα δεδομένων \textbf{χωρίς} επισημειώσεις. Τα αποτελέσματα που τονίζονται αφορούν τα καλύτερα σκορ για κάθε σύνολο δεδομένων ως προς την μέση τιμή. Στην περίπτωση που δεν σημειώνεται η διακύμανση, είναι διότι αντιστοιχεί σε μηδέν με ακρίβεια δύο δυαδικών ψηφίων. Με \en{OUT-OF-MEMORY} συβολίζουμε την περίπτωση που ο υπολογισμός του πυρήνα ξεπέρασε το όριο μνήμης που είχαμε θέσει, και με \en{TIMEOUT} το όριο χρόνου.}
\label{tab:acc:unlabelled}
\end{table}


\begin{table}[]
\centering
\begin{adjustbox}{angle=90}
\resizebox{1.0\textheight}{!}{
\en{\tiny
\begin{tabular}{|l|c|c|c|c|c|c|c|}
\hline
\cellcolor[HTML]{999903}  &       IMDB-BINARY &        IMDB-MULTI &            COLLAB &   REDDIT-MULTI-12K &     REDDIT-BINARY &    REDDIT-MULTI-5K \\\hline
GH              &            2.19 m &            2.06 m &            5.86 h &            TIMEOUT &            TIMEOUT &            TIMEOUT \\\hline
GR              &  22.76 m ± 6.74 m &  21.74 m ± 5.41 m &  2.97 h ± 38.08 m &  43.44 m ± 12.12 m &   44.08 m ± 6.84 m &  44.11 m ± 16.82 m \\\hline
$\text{L}_{\vartheta}$   &  5.32 h ± 36.49 m &  6.55 h ± 50.71 m &           TIMEOUT &            TIMEOUT &            TIMEOUT &            TIMEOUT \\\hline
ML              &   1.37 h ± 17.3 m &  1.69 h ± 28.28 m &             9.4 h &                  - &             \cemph{8.36 h} &            47.87 m \\\hline
NH              &  \cemph{21.83 s} ± 2.37 s &  \cemph{26.07 s} ± 6.78 s &  \cemph{35.83 m} ± 5.62 m &    9.06 h ± 1.59 h &   19.46 m ± 2.27 m &   \cemph{2.76 h} ± 14.29 m \\\hline
NSPDK           &   4.3 m ± 27.55 s &  2.82 m ± 27.77 s &           TIMEOUT &            TIMEOUT &            TIMEOUT &            TIMEOUT \\\hline
ODD-STh         &   4.47 s ± 0.66 s &   4.85 s ± 0.45 s &            2.02 h &             8.34 m &             1.89 m &             4.82 m \\\hline
P2K             &    7.41 s ± 0.6 s &  14.26 s ± 0.51 s &   4.57 m ± 7.38 s &  20.28 m ± 49.64 s &      1.4 m ± 2.2 s &   5.87 m ± 16.83 s \\\hline
PM              &   1.47 m ± 6.36 s &  2.22 m ± 14.39 s &  36.44 m ± 2.56 m &             \cemph{3.84 h} &  10.09 m ± 16.68 s &            51.75 m \\\hline
RW              &  7.35 m ± 14.66 s &  13.68 m ± 8.94 s &           13.64 h &            TIMEOUT &            TIMEOUT &            TIMEOUT \\\hline
SM              &           TIMEOUT &           TIMEOUT &           TIMEOUT &                  - &                  - &                  - \\\hline
SP              &           11.51 s &            7.92 s &            1.15 h &            TIMEOUT &              4.8 h &            12.67 h \\\hline
$\text{SVM}_{\vartheta}$ &   39.4 s ± 4.68 s &  1.01 m ± 10.57 s &  5.96 m ± 53.52 s &   52.06 m ± 2.71 m &   19.47 m ± 4.82 s &  23.13 m ± 41.32 s \\\hline
VH              &            0.07 s &            0.15 s &            1.12 s &             6.37 s &             0.67 s &              2.2 s \\\hline
\end{tabular}
}
}
\end{adjustbox}
\caption[Μέσοι όροι και διακυμάνσεις των χρόνων εκτέλεσης που αντιστοιχούν στις καλύτερες τιμές της μετρικής ευστοχίας για $10$ επαναλήψεις \en{$10$-fold cross validation} στα σύνολα δεδομένων χωρίς επισημειώσεις.]{\small Μέσοι όροι και διακυμάνσεις των χρόνων εκτέλεσης που αντιστοιχούν στις καλύτερες τιμές της μετρικής ευστοχίας για $10$ επαναλήψεις \en{$10$-fold cross validation} στα σύνολα δεδομένων \textbf{χωρίς} επισημειώσεις, όπως καταγράφονται στον πίνακα \ref{tab:acc:unlabelled}. Οι χρόνοι που τονίζονται αφορούν αυτούς με τα καλύτερα σκορ ευστοχίας. Τα κελιά που σημειώνονται με ``-'' αφορούν τιμές που η εκτέλεση δεν ολοκληρώθηκε, καθώς υπερέβη την μέγιστη επιτρεπτή μνήμη και με \en{``TIMEOUT''} όταν ξεπέρασε το μέγιστο επιτρεπτό χρόνο.}
\label{tab:time:unlabelled}
\end{table}

\begin{table}[]
\centering
\begin{adjustbox}{angle=90}
\resizebox{1.0\textheight}{!}{
\en{\tiny
\begin{tabular}{|l|c|c|c|c|c|c|c|}
\hline
\cellcolor[HTML]{4C388A} &     IMDB-BINARY &     IMDB-MULTI &           COLLAB & REDDIT-MULTI-12K &   REDDIT-BINARY & REDDIT-MULTI-5K \\\hline
GH              &           0.25G &           0.28G &           11.98G &                - &               - &               - \\\hline
GR              &   0.22G ± 0.19M &   0.22G ± 2.15M &  11.16G ± 10.12M &    8.22G ± 2.22M &   1.18G ± 5.62M &    3.31G ± 6.4M \\\hline
$\text{L}_{\vartheta}$   &    1.32G ± 0.1M &    0.36G ± 0.1M &                - &                - &               - &               - \\\hline
ML              &    0.3G ± 4.84M &   0.32G ± 7.35M &           12.75G &       OUT-OF-MEMORY &          \cemph{18.27G} &          42.64G \\\hline
NH              &   \cemph{0.31G} ± 0.43M &   \cemph{0.34G} ± 0.83M &  \cemph{22.28G} ± 13.59M &  15.65G  ± 0.13G &   2.4G ± 19.04M &   \cemph{7.5G} ± 45.81M \\\hline
NSPDK           &   0.42G ± 7.02M &  0.54G ± 21.74M &                - &                - &               - &               - \\\hline
ODD-STh         &   0.31G ± 5.76M &   0.34G ± 4.27M &            49.4G &           14.41G &           2.17G &           6.68G \\\hline
P2K             &   0.25G ± 0.04M &   0.26G ± 0.11M &   11.99G ± 0.58M &   44.53G ± 3.76M &  10.13G ± 1.91M &  25.29G ± 3.64M \\\hline
PM              &   0.21G ± 0.65M &   0.23G ± 0.99M &   11.12G ± 0.03M &             \cemph{8.6G} &   1.17G ± 0.79M &           3.57G \\\hline
RW              &  0.37G  ± 0.13G &  0.23G ± 10.08M &           11.31G &       OUT-OF-MEMORY &               - &               - \\\hline
SM              &               - &               - &                - &       OUT-OF-MEMORY &      OUT-OF-MEMORY &      OUT-OF-MEMORY \\\hline
SP              &           0.21G &           0.21G &           11.02G &                - &           1.36G &           3.33G \\\hline
$\text{SVM}_{\vartheta}$ &   0.21G ± 0.21M &    0.22G ± 0.3M &   11.13G ± 0.05M &    8.27G ± 2.49M &   1.35G ± 0.03M &    3.4G ± 0.15M \\\hline
VH              &           0.34G &           0.43G &           22.12G &           14.22G &           2.12G &           6.68G \\\hline
\end{tabular}
}
}
\end{adjustbox}
\caption[Μέσοι όροι και διακυμάνσεις της μέγιστης τιμής μνήμης από τις εκτελέσεις που αντιστοιχούν στις καλύτερες τιμές της μετρικής ευστοχίας για $10$ επαναλήψεις \en{$10$-fold cross validation} στα σύνολα δεδομένων χωρίς επισημειώσεις.]{\small Μέσοι όροι και διακυμάνσεις της μέγιστης τιμής μνήμης από τις εκτελέσεις που αντιστοιχούν στις καλύτερες τιμές της μετρικής ευστοχίας για $10$ επαναλήψεις \en{$10$-fold cross validation} στα σύνολα δεδομένων \textbf{χωρίς} επισημειώσεις όπως φαίνονται στον πίνακα \ref{tab:acc:unlabelled}. Οι χρόνοι που τονίζονται αφορούν αυτούς με τα καλύτερα σκορ ευστοχίας. Τα κελιά που σημειώνονται με ``-'' αφορούν τιμές που η εκτέλεση διακόπηκε καθώς υπερέβη τον μέγιστο επιτρεπτό, ενώ με \en{``OUT-OF-MEMORY''} όταν υπερέβη την μέγιστη επιτρεπτή μνήμη.}
\label{tab:mem:unlabelled}
\end{table}

\newpage
\subsection{Με Διακριτές Επισημειώσεις}
Στα σύνολα δεδομένων \en{NCI1, PTC\_MR, ENZYMES, DD, PROTEINS, MUTAG} και \en{AIDS} με διακριτές επισημειώσεις, όπως περιγράφονται στην υποενότητα \label{ssec:lab}, εκτελέστηκαν οι πυρήνες \en{AIDS, GH, ML, NH, NSPDK ODD-STh, PK, PM, RW, SM, SP, VH, CORE-SP, CORE-WL-VH, WL-PM, WL-SP} και \en{WL-VH}.
Καλύτερη επίδοση τόσο σε ευστοχία όσο και σε χρόνο και μνήμη, είχαν οι πυρήνες \en{WL-PM, CORE-SP} και \en{GH}, με τον πρώτο να παρουσιάζει καλύτερη επίδοση στο \en{NCI1, PTC\_MR, ENZYMES, MUTAG}, τον δεύτερο στο \en{PROTEINS} και το \en{DD}, ενώ τον τρίτο στο \en{GH} \ref{tab:acc:labelled}.
Κάτι τέτοιο φαίνεται να επαληθεύει την υπόθεση ότι οι παραπάνω σκελετοί πυρήνα κάνουν πιο εκφραστικούς τους υπάρχοντες πυρήνες γράφων είτε εκφράζοντας την ίδια την δομή μέσω των επισημειώσεων (όπως στην περίπτωση του \en{WL}), είτε εισάγοντας μία ιεραρχία στη δομή (όπως στην περίπτωση του \en{CORE}).
Στην πρώτη περίπτωση, χρησιμοποιώντας ένα πυρήνα που εξετάζει τον γράφο σε πολλά επίπεδα όπως ο \en{PM}, σε συνδυασμό με ένα πυρήνα που αποτυπώνει σταδιακά την δομή στις επισημειώσεις, φαίνεται να παράγουμε ένα μέτρο ομοιότητας που διαχωρίζει αποτελεσματικά \textit{υφέρπουσα} ομοιότητα στην δομή\footnote{Κάτι τέτοιο φαίνεται να λειτουργεί σαν τον συνδυασμό δύο μεγεθυντικών φακών σε ένα μικροσκόπιο, που κινώντας σταδιακά τον δεύτερο σε σχέση με τον πρώτο μπορούμε να πετύχουμε ακόμα πιο λεπτές εστιακές δυνατότητες από έναν και μόνο φακό, διακρίνοντας τελικά ακόμα μεγαλύτερη λεπτομέρεια στο υπό εξέταση αντικείμενο.}.
Από την άλλη, η χρήση του \en{Weisfeiler-Lehman} φαίνεται να αυξάνει πολύ τον χώρο που απαιτείται σε σχέση με τον ίδιο τον πυρήνα \en{PM} γεγονός που φαίνεται να σχετίζεται με την ανάγκη αποθήκευσης των πυρήνων \en{PM} που έχουν γίνει \en{fit} προς μελλοντική χρήση από τον χρήστη της βιβλιοθήκης, σε συνδυασμό με τα πλήθος των επιπέδων που πετυχαίνουν το μέγιστο σκορ, πράγμα που μπορεί να φανεί και από τους χρόνους εκτέλεσης.
Όσον αφορά τον συνδυασμό \en{CORE-SP}, η διαχωριστική πληροφορία που κωδικοποιούν οι τιμές των ελάχιστων μονοπατιών φαίνεται να οργανώνεται μέσω των \textit{κορών} με τρόπο που οι ασυνάρτητες συμπτώσεις στις τιμές τους, χάνουν την επιρροή τους στο τελικό αποτέλεσμα.
Ταυτόχρονα η μνήμη δεν αυξάνεται σημαντικά, χωρίς κάτι τέτοιο να μην συμβαίνει στην περίπτωση του χρόνου, όπως φαίνεται στο σύνολο δεδομένων \en{DD}.
Η μέγιστη δυνατή ευστοχία του \en{GH} στο \en{AIDS}, δεν φαίνεται ιδιαίτερα σημαντική σε σχέση με τα σκορ των υπολοίπων πυρήνων.

\begin{table}[]
\centering
\begin{adjustbox}{angle=90}
\resizebox{1.0\textheight}{!}{
\en{
\begin{tabular}{|l|c|c|c|c|c|c|c|}
\hline
\cellcolor[HTML]{CB0000} & NCI1         & PTC\_MR      & ENZYMES      & DD           & PROTEINS     & MUTAG        & AIDS        \\\hline
GH         &  71.36 ± 0.13 &  55.64 ± 2.03 &  36.47 ± 2.13 &       TIMEOUT &  74.19 ± 0.42 &  82.11 ± 2.13 &  \cemph{99.57} ± 0.02 \\\hline
ML         &   79.4 ± 0.47 &  59.95 ± 1.71 &  53.08 ± 1.53 &  78.28 ± 0.99 &  73.89 ± 0.93 &   86.11 ± 1.6 &  98.48 ± 0.12 \\\hline
NH         &  74.81 ± 0.37 &    60.5 ± 2.1 &  43.43 ± 1.45 &  76.02 ± 0.94 &   75.55 ± 1.0 &  87.74 ± 1.17 &  99.54 ± 0.02 \\\hline
NSPDK      &  74.36 ± 0.31 &  60.04 ± 1.15 &  41.97 ± 1.66 &  78.76 ± 0.56 &  73.17 ± 0.76 &  82.46 ± 1.55 &   98.04 ± 0.2 \\\hline
ODD-STh    &  75.03 ± 0.45 &  59.08 ± 1.85 &  31.87 ± 1.35 &  75.82 ± 0.54 &  70.49 ± 0.64 &  79.01 ± 2.04 &   90.75 ± 0.3 \\\hline
P2K        &  82.12 ± 0.22 &   59.3 ± 1.24 &  44.48 ± 1.63 &  78.43 ± 0.55 &  72.71 ± 0.62 &  77.23 ± 1.22 &  96.51 ± 0.38 \\\hline
PM         &  73.11 ± 0.49 &  57.99 ± 2.45 &  42.67 ± 1.78 &  76.98 ± 0.84 &   71.9 ± 0.79 &  84.72 ± 1.67 &  99.56 ± 0.08 \\\hline
RW         &       TIMEOUT &   51.26 ± 2.3 &   12.9 ± 1.42 &       OUT-OF-MEMORY &  69.31 ± 0.29 &  82.24 ± 2.87 &  79.52 ± 0.58 \\\hline
SM         &       TIMEOUT &  57.91 ± 1.73 &   35.68 ± 0.8 &       OUT-OF-MEMORY &       OUT-OF-MEMORY &  84.04 ± 1.55 &  91.96 ± 0.18 \\\hline
SP         &  72.25 ± 0.28 &  59.26 ± 2.34 &  40.13 ± 1.34 &  78.93 ± 0.53 &  75.92 ± 0.35 &   82.54 ± 1.0 &  99.41 ± 0.12 \\\hline
VH         &  56.09 ± 0.35 &  58.09 ± 0.62 &  16.87 ± 1.56 &   74.83 ± 0.4 &  70.93 ± 0.28 &  71.87 ± 1.83 &  79.78 ± 0.13 \\\hline
WL-PM      &  \cemph{85.31} ± 0.42 &  \cemph{64.52} ± 1.36 &  \cemph{57.72} ± 0.84 &       OUT-OF-MEMORY &  75.63 ± 0.49 &   \cemph{88.6} ± 0.95 &  99.37 ± 0.04 \\\hline
WL-SP      &  61.43 ± 0.32 &  55.51 ± 1.68 &   28.23 ± 1.0 &  75.66 ± 0.42 &  71.88 ± 0.22 &  82.29 ± 1.93 &  99.36 ± 0.02 \\\hline
WL-VH      &   85.03 ± 0.2 &  63.28 ± 1.34 &  53.15 ± 1.22 &  78.88 ± 0.46 &  75.45 ± 0.33 &   84.0 ± 1.25 &  98.51 ± 0.05 \\\hline
CORE-SP    &  73.87 ± 0.19 &  58.21 ± 1.87 &  41.55 ± 1.66 &  \cemph{79.33} ± 0.65 &   \cemph{76.31} ± 0.4 &  85.13 ± 2.46 &  99.47 ± 0.05 \\\hline
CORE-WL-VH &  85.12 ± 0.21 &  63.03 ± 1.67 &  52.37 ± 1.29 &   78.91 ± 0.5 &  75.46 ± 0.38 &   85.9 ± 1.44 &   98.7 ± 0.09 \\\hline
\end{tabular}
}
}
\end{adjustbox}
\caption[Μέσοι όροι και διακυμάνσεις της μετρικής ευστοχίας για $10$ επαναλήψεις  \en{$10$-fold cross validation} στα σύνολα δεδομένων με διακριτές επισημειώσεις.]{\small Μέσοι όροι και διακυμάνσεις της μετρικής ευστοχίας για $10$ επαναλήψεις  \en{$10$-fold cross validation} στα σύνολα δεδομένων με \textbf{διακριτές} επισημειώσεις. Τα αποτελέσματα που τονίζονται αφορούν τα καλύτερα σκορ για κάθε σύνολο δεδομένων ως προς την μέση τιμή. Με \en{OUT-OF-MEMORY} περιγράφουμε την περίπτωση που ο υπολογισμός του πυρήνα ξεπέρασε το όριο μνήμης που είχαμε θέσει, και με \en{TIMEOUT} το όριο χρόνου.}
\label{tab:acc:labelled}
\end{table}

\begin{table}[]
\centering
\begin{adjustbox}{angle=90}
\resizebox{1.0\textheight}{!}{
\en{\tiny
\begin{tabular}{|l|c|c|c|c|c|c|c|}
\hline
\cellcolor[HTML]{999903}       & NCI1              & PTC\_MR         & ENZYMES           & DD               & PROTEINS         & MUTAG            & AIDS             \\\hline
GH         &    3.75 h ± 1.77 m &   1.57 m ± 0.67 s &   15.64 m ± 1.31 s &           TIMEOUT &   3.72 h ± 2.51 m &   24.7 s ± 0.11 s &  \cemph{38.86 m} ± 3.21 s \\\hline
ML         &    5.52 h ± 1.24 h &  19.37 m ± 6.61 m &  56.73 m ± 13.52 m &   3.68 h ± 1.81 h &  2.34 h ± 50.11 m &  10.05 m ± 2.44 m &    1.2 h ± 16.4 m \\\hline
NH         &   7.08 m ± 50.97 s &   1.31 s ± 0.21 s &   11.17 s ± 2.14 s &  6.29 m ± 52.25 s &  41.81 s ± 9.15 s &    0.4 s ± 0.02 s &   33.3 s ± 1.54 s \\\hline
NSPDK      &    6.16 m ± 1.22 m &   7.66 s ± 1.19 s &    27.02 s ± 2.8 s &   4.61 h ± 1.15 h &    9.16 m ± 1.8 m &   4.05 s ± 0.93 s &  1.21 m ± 22.32 s \\\hline
ODD-STh    &   46.04 m ± 2.34 m &   4.03 s ± 0.64 s &   50.05 s ± 7.34 s &  27.99 m ± 4.76 m &   4.13 m ± 1.03 m &   1.54 s ± 0.18 s &   2.09 m ± 5.48 s \\\hline
P2K        &   10.46 m ± 6.56 s &    1.81 s ± 0.3 s &   12.05 s ± 0.95 s &   9.57 m ± 1.46 m &   51.2 s ± 7.22 s &   0.48 s ± 0.07 s &   1.73 m ± 6.79 s \\\hline
PM         &   37.62 m ± 1.93 m &   11.35 s ± 1.9 s &   31.38 s ± 2.52 s &  5.81 m ± 36.39 s &  1.45 m ± 10.62 s &   2.59 s ± 0.23 s &    2.8 m ± 17.2 s \\\hline
RW         &            TIMEOUT &   6.69 m ± 1.85 m &     4.4 h ± 2.05 h &                 - &           51.17 m &  1.78 m ± 18.48 s &            1.87 h \\\hline
SM         &            TIMEOUT &            4.33 m &             3.43 h &                 - &                 - &            1.95 m &            4.45 h \\\hline
SP         &             1.16 m &            1.52 s &            11.03 s &           55.98 m &            1.32 m &            0.92 s &           13.93 s \\\hline
VH         &             0.84 s &            0.02 s &             0.04 s &            0.24 s &             0.1 s &            0.01 s &            0.25 s \\\hline
WL-PM      &  \cemph{13.53 h} ± 43.83 m &  \cemph{11.14 m} ± 3.66 m &   \cemph{1.09 h} ± 23.07 m &                 - &   5.62 h ± 1.07 h &   \cemph{3.7 m} ± 41.35 s &    5.92 h ± 1.2 h \\\hline
WL-SP      &            15.49 m &           12.55 s &             1.45 m &            7.46 h &            8.06 m &            7.02 s &            1.56 m \\\hline
WL-VH      &   7.09 m ± 26.75 s &   0.55 s ± 0.03 s &    3.81 s ± 0.37 s &  5.88 m ± 29.96 s &  32.48 s ± 3.09 s &   0.21 s ± 0.01 s &   40.49 s ± 0.8 s \\\hline
CORE-SP    &             3.28 m &            3.97 s &            48.02 s &            \cemph{5.04 h} &            \cemph{3.53 m} &            2.69 s &           40.11 s \\\hline
CORE-WL-VH &   14.51 m ± 2.21 m &   1.05 s ± 0.05 s &   12.52 s ± 0.97 s &           17.04 m &   1.28 m ± 6.55 s &   0.55 s ± 0.03 s &  54.79 s ± 7.41 s \\\hline
\end{tabular}
}
}
\end{adjustbox}
\caption[Μέσοι όροι και διακυμάνσεις των χρόνων εκτέλεσης που αντιστοιχούν στις καλύτερες τιμές της μετρικής ευστοχίας για $10$ επαναλήψεις \en{$10$-fold cross validation} στα σύνολα δεδομένων με διακριτές επισημειώσεις.]{\small Μέσοι όροι και διακυμάνσεις των χρόνων εκτέλεσης που αντιστοιχούν στις καλύτερες τιμές της μετρικής ευστοχίας για $10$ επαναλήψεις \en{$10$-fold cross validation} στα σύνολα δεδομένων με \textbf{διακριτές} επισημειώσεις, όπως καταγράφονται στον πίνακα \ref{tab:acc:labelled}. Οι χρόνοι που τονίζονται αφορούν αυτούς με τα καλύτερα σκορ ευστοχίας. Τα κελιά που σημειώνονται με ``-'' αφορούν τιμές που η εκτέλεση δεν ολοκληρώθηκε, καθώς υπερέβη την μέγιστη επιτρεπτή μνήμη και με \en{``TIMEOUT''} όταν ξεπέρασε το μέγιστο επιτρεπτό χρόνο.}
\label{tab:time:labelled}
\end{table}

\begin{table}[]
\centering
\begin{adjustbox}{angle=90}
\resizebox{1.0\textheight}{!}{
\en{\tiny
\begin{tabular}{|l|c|c|c|c|c|c|c|}
\hline
\cellcolor[HTML]{4C388A} & NCI1            & PTC\_MR        & ENZYMES        & DD              & PROTEINS        & MUTAG          & AIDS           \\\hline
GH         &    3.06G ± 8.39M &   0.15G ± 0.24M &   0.41G ± 0.02M &                - &    1.73G ± 0.01M &           0.12G &    \cemph{0.89G} ± 0.86M \\\hline
ML         &   1.15G ± 17.07M &   0.13G ± 3.52M &  0.24G ± 10.67M &   5.68G ± 59.93M &   0.41G ± 17.92M &   0.12G ± 1.11M &    0.34G ± 5.06M \\\hline
NH         &   0.76G ± 13.23M &   0.11G ± 0.13M &   0.18G ± 0.93M &   1.88G ± 10.56M &     0.3G ± 1.56M &   0.11G ± 0.02M &    0.24G ± 0.73M \\\hline
NSPDK      &    3.9G  ± 0.38G &   0.16G ± 5.48M &  0.34G ± 16.04M &   17.64G  ± 3.8G &   1.33G  ± 0.15G &   0.13G ± 2.96M &   1.07G  ± 0.15G \\\hline
ODD-STh    &  33.71G  ± 2.04G &  0.18G ± 12.29M &  1.18G  ± 0.15G &  25.95G  ± 4.16G &    4.0G  ± 0.91G &    0.12G ± 2.6M &    2.12G ± 76.9M \\\hline
P2K        &    1.34G ± 0.33M &   0.12G ± 0.38M &   0.19G ± 0.62M &   6.54G  ± 0.26G &    0.32G ± 0.99M &   0.11G ± 0.73M &     0.4G ± 0.58M \\\hline
PM         &   0.99G ± 32.82M &   0.12G ± 4.59M &   0.19G ± 0.96M &   2.07G ± 16.26M &     0.3G ± 1.04M &   0.11G ± 0.64M &    0.28G ± 8.25M \\\hline
RW         &                - &  0.33G ± 89.39M &  2.67G  ± 1.11G &       OUT-OF-MEMORY &            0.45G &   0.11G ± 1.89M &            0.33G \\\hline
SM         &                - &           0.13G &           1.14G &       OUT-OF-MEMORY &       OUT-OF-MEMORY &           0.11G &             0.4G \\\hline
SP         &            0.81G &           0.11G &           0.18G &            2.84G &            0.29G &           0.11G &            0.29G \\\hline
VH         &            0.71G &           0.11G &           0.19G &            1.83G &            0.33G &            0.1G &            0.33G \\\hline
WL-PM      &   \cemph{16.76G}  ± 0.8G &  \cemph{2.56G}  ± 0.88G &  \cemph{9.97G}  ± 3.87G &       OUT-OF-MEMORY &  26.71G  ± 4.72G &  \cemph{1.65G}  ± 0.32G &  15.39G  ± 4.58G \\\hline
WL-SP      &            1.85G &           0.12G &           0.19G &            2.12G &            0.33G &           0.11G &            0.51G \\\hline
WL-VH      &   7.44G  ± 0.85G &   0.14G ± 2.08M &  0.45G ± 34.74M &  18.94G  ± 1.48G &   1.79G  ± 0.14G &   0.12G ± 0.32M &    1.73G ± 28.3M \\\hline
CORE-SP    &            1.44G &           0.12G &            0.2G &            \cemph{8.98G} &            \cemph{0.38G} &           0.11G &            0.43G \\\hline
CORE-WL-VH &  16.72G  ± 2.23G &   0.16G ± 4.25M &  1.11G ± 82.84M &           50.41G &   4.67G  ± 0.41G &   0.12G ± 1.86M &   2.29G  ± 0.25G \\\hline
\end{tabular}
}
}
\end{adjustbox}
\caption[Μέσοι όροι και διακυμάνσεις της μέγιστης τιμής μνήμης από τις εκτελέσεις που αντιστοιχούν στις καλύτερες τιμές της μετρικής ευστοχίας για $10$ επαναλήψεις \en{$10$-fold cross validation} στα σύνολα δεδομένων με διακριτές επισημειώσεις.]{\small Μέσοι όροι και διακυμάνσεις της μέγιστης τιμής μνήμης από τις εκτελέσεις που αντιστοιχούν στις καλύτερες τιμές της μετρικής ευστοχίας για $10$ επαναλήψεις \en{$10$-fold cross validation} στα σύνολα δεδομένων με \textbf{διακριτές} επισημειώσεις όπως φαίνονται στον πίνακα \ref{tab:acc:labelled}. Οι χρόνοι που τονίζονται αφορούν αυτούς με τα καλύτερα σκορ ευστοχίας. Τα κελιά που σημειώνονται με ``-'' αφορούν τιμές που η εκτέλεση διακόπηκε καθώς υπερέβη τον μέγιστο επιτρεπτό, ενώ με \en{``OUT-OF-MEMORY''} όταν υπερέβη την μέγιστη επιτρεπτή μνήμη.}
\label{tab:mem:labelled}
\end{table}\newpage
\subsection{Με Επισημειώσεις Χαρακτηριστικών}
Στα σύνολα δεδομένων \en{ENZYMES}, \en{SYNTHETICnew}, \en{Synthie}, \en{BZR} και \en{PROTEINS\_full} με επισημειώσεις χαρακτηριστικών, όπως περιγράφονται στην υποενότητα \label{ssec:atr}, εκτελέστηκαν οι πυρήνες \en{GH}, \en{ML}, \en{PK}, \en{SM}, \en{SP}.
Την καλύτερη επίδοση ευστοχίας σε όλα τα σύνολα δεδομένων, εμφάνισε ο πυρήνες \en{GH}, όπως φαίνεται στον πίνακα \ref{tab:acc:attributed}.
Ταυτόχρονα ο πυρήνας \en{ML} φαίνεται να τον πλησιάζει στα περισσότερα σύνολα δεδομένων, ενώ αξίζει να σημειώσουμε ότι ο χαμηλός χρόνος που φαίνεται να παρουσιάζει, οφείλεται στο γεγονός ότι το καλύτερο αποτέλεσμα επιτυγχάνεται στην περίπτωση όπου $L=0$.
Κάτι τέτοιο φαίνεται να συμβαίνει, λόγω του προβλήματος της υπεροχής της διαγωνίου (\en{diagonal dominance}) της μήτρας πυρήνα, που δεν εμφανίζεται στην περίπτωση που τα διανύσματα χαρακτηριστικών προκύπτουν ως \en{\textit{one-hot vectors}} των διακριτών επισημειώσεων.
Εν συνεχεία, αιτία του παραπάνω φαίνεται να είναι και το γεγονός ότι π.χ. η πληροφορία θέσης των μορίων στο σύνολο δεδομένων \en{ENZYMES} είναι πολύ \textit{συγκεκριμένη}, οδηγώντας στην έντονη ελάττωση της ομοιότητας μεταξύ δύο διαφορετικών γράφων (σε σχέση με τον εαυτό τους) καθώς ο πυρήνας συνυπολογίζει νέες κλίμακες.
Ακόμα ο πυρήνας \en{SM} (που στην περίπτωση αυτή φέρει την συνάρτηση εσωτερικού γινομένου για να υπολογίσει μία τιμή πυρήνα μεταξύ των χαρακτηριστικών στους κόμβους) εκτελείται φέροντας ικανοποιητικό σκορ μόνο στο \en{BZR} ενώ σε όλους τους υπολοίπους διακόπτεται είτε επειδή ξεπερνάει το όριο χρόνου είτε γιατί ξεπερνάει την μνήμη.
Τέλος, ο πυρήνας κοντινότερων μονοπατιών στην περίπτωση των χαρακτηριστικών είναι εξαιρετικά αργός και ως αποτέλεσμα δεν μας ξαφνιάζει η ανεπάρκεια τερματισμού του σε ικανοποιητικό χρόνο σε όλα τα σύνολα δεδομένων.\par
Όσον αφορά την μνήμη και τους χρόνους, βλέπουμε πως παρότι ο πυρήνας \en{GH} είναι ικανοποιητικός παρουσιάζει μεγάλο \en{overhead} όσον αφορά το σύνολο δεδομένων \en{PROTEINS\_full} τόσο σε μνήμη όσο και σε χώρο παρά τα βέλτιστα αποτελέσματά του.
\newpage
\begin{table}[ht!]
\centering
\begin{adjustbox}{angle=0}
\resizebox{1.0\textwidth}{!}{
\en{
\begin{tabular}{|l|c|c|c|c|c|}
\hline
\cellcolor[HTML]{CB0000}  & ENZYMES     & SYNTHETICnew & Synthie     & BZR         & PROTEINS\_full \\\hline
GH  &  \cemph{66.25} ± 1.24 &  \cemph{76.43} ± 1.97 &  \cemph{71.75} ± 1.65 &  \cemph{82.58} ± 1.05 &  \cemph{72.49} ± 0.34 \\\hline
ML  &  65.55 ± 0.93 &   47.9 ± 2.13 &  69.42 ± 1.98 &  82.33 ± 1.29 &  70.55 ± 0.99 \\\hline
P2K &   15.42 ± 1.0 &   47.9 ± 3.26 &   48.9 ± 2.05 &  78.76 ± 0.02 &  59.56 ± 0.01 \\\hline
SM  &       TIMEOUT &       TIMEOUT &       TIMEOUT &  80.52 ± 0.43 &       OUT-OF-MEMORY \\\hline
SP  &       TIMEOUT &       TIMEOUT &       TIMEOUT &       TIMEOUT &       TIMEOUT \\\hline
\end{tabular}
}
}
\end{adjustbox}
\caption[Μέσοι όροι και διακυμάνσεις της μετρικής ευστοχίας για $10$ επαναλήψεις  \en{$10$-fold cross validation} στα σύνολα δεδομένων με συνεχείς επισημειώσεις.]{\small Μέσοι όροι και διακυμάνσεις της μετρικής ευστοχίας για $10$ επαναλήψεις  \en{$10$-fold cross validation} στα σύνολα δεδομένων με \textbf{συνεχείς} επισημειώσεις. Τα αποτελέσματα που τονίζονται αφορούν τα καλύτερα σκορ για κάθε σύνολο δεδομένων ως προς την μέση τιμή. Με \en{OUT-OF-MEMORY} συμβολίζουμε την περίπτωση που ο υπολογισμός του πυρήνα ξεπέρασε το όριο μνήμης που είχαμε θέσει, και με \en{TIMEOUT} το όριο χρόνου.}
\label{tab:acc:attributed}
\end{table}

\begin{table}[ht!]
\centering
\begin{adjustbox}{angle=0}
\resizebox{1.0\textwidth}{!}{
\en{\footnotesize
\begin{tabular}{|l|c|c|c|c|c|}
\hline
\cellcolor[HTML]{999903}   & ENZYMES          & SYNTHETICnew     & Synthie           & BZR            & PROTEINS\_full \\\hline
GH  &           \cemph{16.61} m &  \cemph{13.91} m ± 3.84 s &          \cemph{24.37} m &  \cemph{4.41} m ± 0.37 s &           \cemph{5.28} h \\\hline
ML  &           20.54 s &            6.64 s &          11.87 s &           8.76 s &           1.43 m \\\hline
P2K &  15.91 s ± 1.82 s &   14.31 s ± 1.8 s &  30.6 s ± 6.45 s &           10.4 s &  1.74 m ± 3.39 s \\\hline
SM  &           TIMEOUT &           TIMEOUT &          TIMEOUT &           8.03 h &                - \\\hline
SP  &           TIMEOUT &           TIMEOUT &          TIMEOUT &          TIMEOUT &          TIMEOUT \\\hline
\end{tabular}
}
}
\end{adjustbox}
\caption[Μέσοι όροι και διακυμάνσεις των χρόνων εκτέλεσης που αντιστοιχούν στις καλύτερες τιμές της μετρικής ευστοχίας για $10$ επαναλήψεις \en{$10$-fold cross validation} στα σύνολα δεδομένων με συνεχείς επισημειώσεις.]{\small Μέσοι όροι και διακυμάνσεις των χρόνων εκτέλεσης που αντιστοιχούν στις καλύτερες τιμές της μετρικής ευστοχίας για $10$ επαναλήψεις \en{$10$-fold cross validation} στα σύνολα δεδομένων με \textbf{συνεχείς} επισημειώσεις, όπως καταγράφονται στον πίνακα \ref{tab:acc:attributed}. Οι χρόνοι που τονίζονται αφορούν αυτούς με τα καλύτερα σκορ ευστοχίας. Τα κελιά που σημειώνονται με ``-'' αφορούν τιμές που η εκτέλεση δεν ολοκληρώθηκε, καθώς υπερέβη την μέγιστη επιτρεπτή μνήμη και με \en{``TIMEOUT''} όταν ξεπέρασε το μέγιστο επιτρεπτό χρόνο.}
\label{tab:times:attributed}
\end{table}

\begin{table}[ht!]
\centering
\begin{adjustbox}{angle=0}
\resizebox{1.0\textwidth}{!}{
\en{\footnotesize
\begin{tabular}{|l|c|c|c|c|c|}
\hline
\cellcolor[HTML]{4C388A}  & ENZYMES       & SYNTHETICnew  & Synthie        & BZR           & PROTEINS\_full \\\hline
GH  &          \cemph{0.41G} &  \cemph{0.22G} ± 0.02M &         \cemph{0.31G} &  \cemph{0.18G} ± 0.41M &          \cemph{1.74G} \\\hline
ML  &          0.19G &          0.24G &         0.29G &          0.14G &          0.39G \\\hline
P2K &  0.22G ± 4.63M &  0.23G ± 3.87M &  0.3G ± 7.64M &          0.14G &  0.53G ± 6.16M \\\hline
SM  &              - &              - &             - &           0.2G &     OUT-OF-MEMORY \\\hline
SP  &              - &              - &             - &              - &              - \\\hline
\end{tabular}
}
}
\end{adjustbox}
\caption[Μέσοι όροι και διακυμάνσεις της μέγιστης τιμής μνήμης από τις εκτελέσεις που αντιστοιχούν στις καλύτερες τιμές της μετρικής ευστοχίας για $10$ επαναλήψεις \en{$10$-fold cross validation} στα σύνολα δεδομένων με συνεχείς επισημειώσεις.]{\small Μέσοι όροι και διακυμάνσεις της μέγιστης τιμής μνήμης από τις εκτελέσεις που αντιστοιχούν στις καλύτερες τιμές της μετρικής ευστοχίας για $10$ επαναλήψεις \en{$10$-fold cross validation} στα σύνολα δεδομένων με \textbf{συνεχείς} επισημειώσεις όπως φαίνονται στον πίνακα \ref{tab:acc:attributed}. Οι χρόνοι που τονίζονται αφορούν αυτούς με τα καλύτερα σκορ ευστοχίας. Τα κελιά που σημειώνονται με ``-'' αφορούν τιμές που η εκτέλεση διακόπηκε καθώς υπερέβη τον μέγιστο επιτρεπτό, ενώ με \en{``OUT-OF-MEMORY''} όταν υπερέβη την μέγιστη επιτρεπτή μνήμη.}
\label{tab:mem:attributed}
\end{table}
%\textcopyright
%\begin{checkhyphens}
%Τι συμφορά, ενώ είσαι καμωμένος
%για τα ωραία και μεγάλα έργα
%η άδικη αυτή σου η τύχη πάντα
%ενθάρρυνσι κ’ επιτυχία να σε αρνείται·
%να σ’ εμποδίζουν ευτελείς συνήθειες,
%και μικροπρέπειες, κι αδιαφορίες.
%Και τι φρικτή η μέρα που ενδίδεις,
%(η μέρα που αφέθηκες κ’ ενδίδεις),
%και φεύγεις οδοιπόρος για τα Σούσα,
%και πηαίνεις στον μονάρχην Aρταξέρξη
%που ευνοϊκά σε βάζει στην αυλή του,
%και σε προσφέρει σατραπείες και τέτοια.
%Και συ τα δέχεσαι με απελπισία
%αυτά τα πράγματα που δεν τα θέλεις.
%Άλλα ζητεί η ψυχή σου, γι’ άλλα κλαίει·
%τον έπαινο του Δήμου και των Σοφιστών,
%τα δύσκολα και τ’ ανεκτίμητα Εύγε·
%την Aγορά, το Θέατρο, και τους Στεφάνους.
%Aυτά πού θα σ’ τα δώσει ο Aρταξέρξης,
%αυτά πού θα τα βρεις στη σατραπεία·
%και τι ζωή χωρίς αυτά θα κάμεις. 
%Επιστημοτεχνολογικίζουσα διαστροφή.
%\end{checkhyphens}
\newpage
\section{Σύνοψη Αποτελεσμάτων}
Η εκτέλεση των παραπάνω πειραμάτων ήταν μία διαδικασία που διήρκεσε περίπου τρεις μήνες.
Μέσα σε αυτό το διάστημα διορθώθηκαν διάφορα λάθη στην υλοποίηση των πυρήνων, ενώ έγινε εκτενής σύγκριση τους με τα αποτελέσματα και τις υλοποιήσεις άλλων πακέτων για την ταυτοποίηση αντιστοιχιών και την διάψευση ύπαρξης προβλημάτων, στην αντίθετη περίπτωση.
Συνολικά τόσο σε ταχύτητα υλοποίησης όσο και σε χρόνο εκτέλεσης οι πυρήνες \en{NH, PM, CORE-SP, WL-PM, ML, GH} φαίνεται να παρουσιάζουν τα καλύτερα αποτελέσματα στο σύνολο των πειραμάτων.
Από την άλλη τόσο το πρόβλημα της \textit{υπεροχής της διαγωνίου} (\en{diagonal dominance}) όσο και το \textit{φράγμα ευστοχίας} (\en{accuracy gap}) φαίνεται να απασχολούν και αυτήν την κατηγορία προβλημάτων, όπου ακόμα και οι φυσικές αναπαραστάσεις πολλών αντικειμένων σε μορφή γράφων δεν είναι ικανές, προκειμένου αυτά να είναι απόλυτα διαχωρίσιμα ή να πετυχαίνουν την απαραίτητη εκφραστική ευστοχία, για μία μέθοδο πυρήνα.
Η καλή επίδοση των σκελετών πυρήνα φαίνεται να συμβαίνει, καθώς η αναπαράσταση των γράφων σε φλοιούς (δομής, συνεκτικότητας, κλπ) ενισχύει την εκφραστικότητα ενός πυρήνα, κλιμακώνοντας (ή υπο-κλιμακώνοντας αντίστοιχα) τις τιμές ομοιότητας, αφού αυτές υπολογίζονται σε ένα καλύτερα οργανωμένο πολλαπλάσιο των δεδομένων.
Παράλληλα οι αλγόριθμοι \en{PM, ML} που έχουν προκύψει από την μεταφορά αλγορίθμων της όρασης υπολογιστών, δίνουν ίσως μία ένδειξη πως η γεωμετρική διάρθρωση μίας εικόνας και η δομή ενός γράφου, ίσως να μπορούν να ειδωθούν υπό το ίδιο ερευνητικό πρίσμα, όσον αφορά την αναγκαιότητα μετάβασης από το τοπικό στο γενικό, κατά την ανάπτυξη νέων πυρήνων.\par
Η αδυναμία των πυρήνων γράφων να ξεπεράσουν το φράγμα ευστοχίας ή αυτό της επικράτησης της διαγωνίου, δεν φαίνεται να είναι αυτό της ανεπάρκειας τους να λύσουν το πρόβλημα του ισομορφισμού.
Είναι αυτό της αδυναμίας τους να \textit{διαχωρίσουν} τις δομές εκείνες που φαίνονται καθοριστικές, για το εκάστοτε πρόβλημα ταξινόμησης.
Από τη στιγμή που αναπαραστάσεις τους στους χώρους \en{Hilbert} δεν είναι πάντα άμεσες, δεν μπορούν σε αρκετές περιπτώσεις  να χρησιμοποιηθούν μέθοδοι όπως αυτές των πολύ δημοφιλών νευρωνικών δικτύων βαθιάς εκμάθησης (\en{deep learning}), μιας και αυτές χρειάζονται τις αναπαραστάσεις όλων των γράφων σε ένα διανυσματικό χώρο (\en{graph embeddings}\footnote{Παράλληλα πολλοί πυρήνες της μορφής των \en{R-frameworks} μπορούν να χρησιμοποιηθούν γι' αυτό το σκοπό μιας και αναπαριστούν κάθε γράφο σε ένα σύνολο χαρακτηριστικών. Από την άλλη η έρευνα σε αυτόν τον χώρο αναπτύσσεται ξεχωριστά, με τα δικά της προβλήματα και τις δικές της προκλήσεις \cite{graphembeddings}.}).
Η μη-επιβλεπόμενη φύση της προσέγγισης τους φαίνεται να είναι και το ίδιο τους το όριο (όσον αφορά τα ίδια τα προβλήματα ταξινόμησης), που ίσως να μην μπορεί να γεφυρωθεί τόσο υπολογιστικά όσο και θεωρητικά.
Ως επακόλουθο κρίνουμε πως η έρευνα στον χώρο των \textbf{επιβλεπόμενων} πυρήνων γράφων, φαίνεται να έχει ιδιαίτερο ερευνητικό ενδιαφέρον.
