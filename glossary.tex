\newcommand{\gloss}[2]{#1 \> \en{#2}\\ }

\chapter*{Γλωσσάριο}
\label{glossary}

\addcontentsline{toc}{chapter}{Γλωσσάριο}

\begin{tabbing}
%τα 'a' ρυθμίζουν το πλάτος των δύο στηλών
  aaaaaaaaaaaaaaaaaaaaaaaaaaaaaaaaaaaaaaaaaaa \= aaaaa\kill
  \Large\textbf{Ελληνικός όρος} \> \Large\textbf{Αγγλικός όρος} \\
  \gloss{αποθετήριο}{repository}
  \gloss{γράφος}{graph}
  \gloss{διανύσματα χαρακτηρηστικών}{feature vectors}
  \gloss{διεπαφή}{interface}
  \gloss{μηχανική μάθηση}{machine learning}
  \gloss{εγχειρίδιο ανάγνωσης}{documentation}
  \gloss{εξόρυξης δεδομένων}{data mining}
  \gloss{επιστημονικού υπολογισμού}{scientific computing}
  \gloss{ισομορφισμός (γράφων)}{graph isomorphism}
  \gloss{κανονική διάταξη}{canonical ordering}
  \gloss{κόμβος}{vertex, node}
  \gloss{μήτρα}{matrix}
  \gloss{πηρύνες γράφων}{graph kernels}
  \gloss{συλλογές δεδομένων}{dataset}
  \gloss{συνδιαστικό}{combinatorial}
  \gloss{συρραφή (κώδικα)}{(code) wrapping}  
  \gloss{συσκευασία (κώδικα)}{(code) packaging}
  \gloss{ταξινομητής μηχανών διανυσμάτων υποστήριξης}{support vector machine classifier}
  \gloss{υψηλό επίπεδο (προγραμματισμού)}{high level (programming)}
  
\end{tabbing}
