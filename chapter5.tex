\chapter{Συμπεράσματα}
\label{chap5}
Λόγω της αύξησης των δεδομένων που εμφανίζονται με αναπαράσταση γράφων, οι υπολογιστικά εφικτές λύσεις σε προβλήματα μηχανικής που εξετάζουν δεδομένα αυτής της μορφής, φαίνονται ιδιαίτερα δελεαστικές.
Λόγω της μεγάλης ανάλυσης που αφορά τις μεθόδους πυρήνα και το εύρος εύρωστων αλγορίθμων που μπορούν να λύσουν προβλήματα μηχανικής μάθησης, αρκετές από τις προσπάθειες επιτέλεσης μηχανικής μάθησης με γράφους εστιάστηκαν στον χώρο των πυρήνων γράφων.
Παρόλο που το πρόβλημα των πυρήνων γράφων είναι ένα πολυμελετημένο πρόβλημα στην σύγχρονη βιβλιογραφία με αρκετές πολύ αποτελεσματικές προσεγγίσεις, δεν είχε επιχειρηθεί στο παρελθόν η συλλογή του σε ένα ώριμο υπολογιστικό πακέτο ευρείας χρήσης σε μία εύκολη στην χρήση γλώσσα προγραμματισμού.
Ως επακόλουθο, επιλέξαμε να σχεδιάσουμε το πακέτο \en{GraKeL} το οποίο περιέχει τους πιο πρωτοποριακούς πυρήνες γράφων που εμφανίζονται στην βιβλιογραφία των τελευταίων χρόνων σε γλώσσα \en{Python}.
Περιέχει συμπληρωματικό εγχειρίδιο χρήσης και διανέμεται ως λογισμικό ελεύθερου κώδικα με συνεχή ενσωμάτωση σε ανοιχτό αποθετήριο στον ιστό.
Ταυτόχρονα η συμβατότητα του με την δημοφιλή βιβλιοθήκη επιστημονικού υπολογισμού \en{scikit-learn}, παρέχει την δυνατότητα εύκολης ενσωμάτωσης υπάρχοντων έργων μηχανικής μάθησης (όπως η ταξινόμηση ή ή συσταδοποίηση), σε δεδομένα εισόδου που φέρουν την μορφή γράφων.
Η πειραματική αξιολόγηση του \en{\texttt{GraKeL}} φαίνεται ικανοποιητική τόσο σε χρόνους όσο και στα αποτελέσματα εν συγκρίσει με πλήθος εργασιών της βιβλιογραφίας που μελετήσαμε.
Από την άλλη το \en{\texttt{GraKeL}} είναι ένα καινούργιο πακέτο που ίσως αλλάξει μέσα στο μέλλον.
Σημαντικό είναι λοιπόν να προτείνουμε μελλοντικές επεκτάσεις.
\section{Μελλοντικές Επεκτάσεις}
Στην συνέχεια θα αναφέρουμε ορισμένα θέματα, τα οποία θα μπορούσαν να είναι αντικείμενο πιθανών επεκτάσεων.
\paragraph*{Η Κλάση \en{\texttt{Graph}}}
Η κλάση \en{\texttt{Graph}} σχεδιάστηκε προκειμένου να στηρίζει αποκλειστικά τους γράφους του πακέτου προσφέροντας ένα μικρό εύρος λειτουργικών συναρτήσεων που είναι κοινοί μεταξύ όλων των πυρήνων.
Λόγω της αρκετά διαφορετικής τους φύσης και της σύνδεσης κάθε πυρήνα με ένα είδος αναπαράστασης γράφου (π.χ. ως λίστα ακμών, ως πίνακας γειτνίασης κλπ) η κλάση αυτή συνδέθηκε με ένα μικρό πλήθος λειτουργιών, που καλύπτουν την ανάγκη ενός γενικού τύπου αναπαράστασης.
Η ευρωστία, η ελαχιστοποίηση του χρόνου και της μνήμης και τέλος η αφαίρεση των λαθών (στο βαθμό που δεν έχει ήδη γίνει) σε αυτήν την κλάση φαίνεται ένα πολύ σημαντικό πρόβλημα.
Έτσι η περαιτέρω μελέτη, ελαχιστοποίηση και υπολογιστική βελτιστοποίηση του συνολικού κώδικα αυτού του αντικειμένου φαίνεται να είναι στη βάση μελλοντικών επεκτάσεων τόσο για την βελτίωση της απόδοσης υπάρχοντων πυρήνων όσο και για την δυνατότητα (εύκολης) ενσωμάτωσης νέων.
\paragraph*{Πυρήνες και με Συνεχείς και Διακριτές Επισημειώσεις}
Αυτήν την στιγμή η είσοδος κάθε αντικειμένου φαίνεται να υποστηρίζει είτε συνεχείς είτε διακριτές επισημειώσεις στους κόμβους και στις ακμές.
Παρόλο που κάθε διακριτή αναπαράσταση επισημειώσεων μπορεί να μετασχηματιστεί σε μία συνεχή, κάτι τέτοιο δεν φαίνεται αποδοτικό για πυρήνες με μεγάλο πλήθος διαφορετικών επισημειώσεων.
Συνεπώς τόσο η μορφή της εισόδου και κατ' επέκταση η κλάση \en{\texttt{Graph}} όσο και οι αντίστοιχοι πυρήνες πρέπει να προσαρμοστούν για να συμπεριλαμβάνουν αυτήν την κοινή πληροφορία, σε υπάρχοντες και μελλοντικούς πυρήνες.
\paragraph*{Πιο Σύνθετοι Σκελετοί Πυρήνα}
Όσον αφορά τους σκελετούς πυρήνα της παρούσας διπλωματικής θεωρήσαμε πως όλοι χρησιμοποιούν στη βάση τους έναν άλλο πυρήνα γράφων.
Κάτι τέτοιο δεν φαίνεται να ισχύει για ένα άλλο πλήθος σκελετών πυρήνα στη βιβλιογραφία, όπως π.χ. οι \en{\textit{deep graph kernels}} \cite{DGK_PINAR} ή οι \en{\textit{optimal assignment kernels}} \cite{kriege_2016_otk}, οι οποίοι χρησιμοποιούν δεδομένα που οι υπάρχοντες πυρήνες παράγουν σε κάποιο ενδιάμεσο στάδιο του υπολογισμού της μήτρας πυρήνα, με βάση το θεωρητικό μοντέλο που ακολουθούν.
Κάτι τέτοιο φυσικά απαιτεί μία πιο αντικειμενοστρεφή οργάνωση των πυρήνων.
\paragraph*{Πιο Αντικειμενοστρεφής Οργάνωση}
Το παρόν λογισμικό δεν έχει οργανωθεί εκτενώς σε μία θεωρητική βάση.
Για παράδειγμα οι πυρήνες δεν διαχωρίζονται με βάση το αν είναι π.χ. \en{R-Convolutional} \cite{vishwanathan2010graph} ή \en{Optimal-Assignment} \cite{Frohlich2005}.
Στην περίπτωση αυτή ένα πλήθος συναρτήσεων που αφορούν λειτουργίες των \en{R-Convolutional} πυρήνων όπως ο υπολογισμός πινάκων χαρακτηριστικών για κάθε σύνολο γράφων (αντί για τιμών πυρήνα), ενώ συμπληρωματικά, στην περίπτωση των \en{Optimal-Assignment}, άλλες λειτουργίες όπως ο υπολογισμός μίας ιεραρχίας ομοιοτήτων μεταξύ όλων των γράφων, μπορούν να οριστούν.
Κατ' επέκταση οι πυρήνες, κληρονομώντας κλάσεις όπως η \en{\texttt{RConvolutional}} ή η \en{\texttt{OptimalAssignment}} κ.ο.κ., θα πρέπει να υλοποιούν αποτελεσματικά ένα εύρος επιπρόσθετων λειτουργιών.
Ερευνητικό ενδιαφέρον παρουσιάζει σε αυτό το σημείο η συνέχιση συμβατότητας με το \en{scikit-learn} και η σχεδίαση όλων των παραπάνω λειτουργιών σε σχέση με την ακολουθία \en{\texttt{fit}}, \en{\texttt{fit\_transform}} και \en{\texttt{transform}}, γεγονός που φαίνεται ιδιαίτερα σημαντικό.
\paragraph*{Περισσότερη Συμβατότητα με τους Πρωτότυπους Κώδικες}
Από την εκτενή μελέτη ενός μεγάλου εύρους της βιβλιογραφίας, διαπιστώσαμε ότι η αποτύπωση της θεωρητικής περιγραφής του υπολογισμού ενός πυρήνα με την ουσιαστική, αυτής δηλαδή που εμφανίζεται σε επίπεδο υλοποίησης, ήταν θολή.
Από μικρές λεπτομέρειες ``που κάνουν την διαφορά'' μέχρι υπολογιστικά μοτίβα που μειώνουν σημαντικά την πολυπλοκότητα, καθώς και τεχνάσματα που αφορούν την ίδια την γλώσσα προγραμματισμού που χρησιμοποιούν για την πρωτότυπη υλοποίηση τους, ικανά να κάνουν τον πυρήνα να φαίνεται πιο αποτελεσματικός στην πράξη από υπάρχοντες.
Η χρήση και η προσαρμογή καίριων σημείων του κώδικα άλλων γλωσσών και η συμβατότητα με τα υπάρχοντα (στο βαθμό που δεν έχει ήδη μελετηθεί) παρουσιάζει ένα ερευνητικό ενδιαφέρον και αποτελεί μία πρόκληση που πρέπει να λύσει κάθε μοντέρνα βιβλιοθήκη επιστημονικού υπολογισμού.\\
\paragraph*{}Εν κατακλείδι, το \en{GraKeL} είναι ένα έργο που η εξέλιξη του φαίνεται να προκαλεί ιδιαίτερο ερευνητικό ενδιαφέρον, ενώ έχει μόλις διανύσει τα πρώτα του στέρεα βήματα.
