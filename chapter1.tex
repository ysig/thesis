\mainmatter
\chapter{Εισαγωγή}
\label{chap1}


\selectlanguage{english}
\selectlanguage{greek}

Η αναπαράσταση γράφων αποτελεί έναν ευέλικτο και περιεκτικό τρόπο με τον οποίο σε διάφορα πεδία της επιστήμης και της τεχνολογίας αναπαρίστανται σύνολα οντοτήτων, ζευγάρια των οποίων συσχετίζονται.
Τα τελευταία χρόνια η αναπαράσταση δεδομένων με την μορφή γράφων έχει γνωρίσει τρομερή άνθιση σε πολλά πεδία, από τα κοινωνικά δίκτυα μέχρι την βιοπληροφορική.
Διάφορα προβλήματα που εγείρουν όλο και πιο πολύ το ενδιαφέρον, εγκαλούν την χρήση τεχνικών μηχανικής μάθησης σε δεδομένα που έχουν αναπαρασταθεί ως γράφοι.
Η δυσκολία και συχνά η αναποτελεσματικότητα των πειραματικών μεθόδων, σε επιστήμες όπως η χημεία και η βιολογία, έχει οδηγήσει στην διερεύνηση τεχνικών στο χώρο της μηχανικής μάθησης, σαν μία πιο αποδοτική εναλλακτική λύση. Για παράδειγμα, η κατανόηση της λειτουργίας μίας πρωτεΐνης με γνωστή ακολουθία μέσα από αναλυτικές πειραματικές μεθόδους, είναι μία ιδιαίτερα επίπονη και χρονοβόρα διαδικασία.
Αντί για αυτό μπορούμε να δοκιμάσουμε υπολογιστικές προσεγγίσεις προκειμένου να προβλέψουμε την πρωτεϊνική λειτουργία.
Αναπαριστώντας τις πρωτεΐνες σαν γράφους, το πρόβλημα μπορεί να οριστεί σαν πρόβλημα ταξινόμησης γράφων (\selectlanguage{english}graph classification problem\selectlanguage{greek}) όπου η λειτουργία μίας νεοανακαλυφθείσας πρωτεΐνης προβλέπεται βάσει ενός συνόλου πρωτεϊνών με γνωστή λειτουργία. 
Παράλληλα με την ανάγκη για μεθόδους μειωμένου υπολογιστικού κόστους, ένα μεγάλο πλήθος εργασιών είναι αδύνατο να έρθουν εις πέρας από ανθρώπους, λόγω του πολύ μεγάλου όγκου δεδομένων που απαιτούν επεξεργασία.
Για παράδειγμα, ο αριθμός των κακόβουλων (\selectlanguage{english}malicious\selectlanguage{greek}) εφαρμογών έχει αυξηθεί αρκετά τα τελευταία χρόνια.
Η χειρονακτική επιθεώρηση δειγμάτων κώδικα με στόχο την ανίχνευση κακόβουλης λειτουργίας δεν είναι εφικτή καθώς ο αριθμός των δειγμάτων αυξάνεται. 
Χάρη στο γεγονός ότι τα περισσότερα καινούργια δείγματα κακόβουλου κώδικα είναι παραλλαγές υπάρχοντος κακόβουλου λογισμικού, μπορούμε αναπαριστώντας τα ως γράφους κλήσης συναρτήσεων (\selectlanguage{english}function call graphs\selectlanguage{greek}) να ανιχνεύσουμε αυτές τις παραλλαγές.
Ως επακόλουθο το πρόβλημα ανίχνευσης κακόβουλου λογισμικού μπορεί να διατυπωθεί ως πρόβλημα ταξινόμησης γράφων, όπου τμήματα κώδικα των οποίων δεν γνωρίζουμε την συμπεριφορά μπορούν να συγκριθούν με κακόβουλα και μη-κακόβουλα δείγματα. 
Από τα παραπάνω γίνεται φανερό ότι η ταξινόμηση γράφων, εξελίσσεται σε ``ζωτικό'' πρόβλημα σε ένα μεγάλο εύρος εφαρμογών.
Παράλληλα η ταξινόμηση γράφων είναι πολύ στενά συνδεδεμένη με το πρόβλημα της σύγκρισης γράφων, ένα κομβικό πρόβλημα στην θεωρία γράφων.
Παρόλο που το πρόβλημα αυτό μελετάται έντονα για πολλά χρόνια, μία γενικά αποδεκτή λύση τόσο σε σχέση με την περιγραφικότητα της όσο και σε σχέση με την αποδοτικότητα της δεν έχει βρεθεί.
Ως επακόλουθο μπορούμε κάλλιστα να συμπεράνουμε ότι μία τέτοια λύση μπορεί να μην υπάρχει, υπόθεση που μας οδηγεί στην αποδοχή της διαφορετικότητας τόσο στον τρόπο απόδοσης όσο και στον βαθμό προσέγγισης των υπαρχόντων μεθόδων στην ολότητα τους.
Παράλληλα η επισήμανση και η επανανοηματοδότηση της έννοιας του υπολογισμού στην σύγχρονη επιστημονικοτεχνολογική ανάπτυξη, οδηγεί στην ανάγκη ύπαρξης υπολογιστικών προτύπων τα οποία θα έχουν οριακά την ίδια θέση την οποία καταλάμβανε το πρότυπο χιλιόγραμμο στην επιστημονικοτεχνολογική ανάπτυξη της εποχής του.
Η δημιουργία μίας βιβλιοθήκης για την επίλυση ενός προβλήματος δεν αποτελεί κατ' αυτόν τον τρόπο μονάχα μία τεχνική λύση σε ένα πρόβλημα, αλλά μία προκείμενη σε ένα επιστημονικό επιχείρημα.
Ταυτόχρονα η εξέλιξη των γλωσσών προγραμματισμού και η επικράτηση του λογισμικού ανοιχτού κώδικα και των ηλεκτρονικών αποθετηρίων στην σύγχρονη ερευνητική πρακτική, έχει δώσει την δυνατότητα οι βιβλιοθήκες να είναι ανοιχτές σε χρήση και σε αλλαγή, με μεγάλη ευκολία, χωρίς φυσικά η τελευταία να μην είναι δέσμια των αντίστοιχων περιορισμών που προκύπτουν με την ίδια την ύπαρξη μίας κοινότητας (όπως η επιστημονική).
Περιοδικά αναθεώρησης (\selectlanguage{english}journals\selectlanguage{greek}), στα οποία οι ερευνήτριες/ητές συνοψίζουν τα αποτελέσματα της ερευνητικής πρακτικής σε σχέση με τα εκάστοτε πεδία τους, συμμετέχοντας έτσι στην διαδικασία ταξινόμησης της γνώσης, έχουν αρχίσει να δίνουν αντίστοιχο χώρο δημοσίευσης στο ίδιο το λογισμικό τοποθετώντας στην διαδικασία επιλογής του κριτήρια εγκυρότητας, διαύγειας, νομικής άδειας χρήσης, δυνατότητα συμμετοχής και προσβασιμότητας στο επίπεδο του κώδικα (βλ. \href{http://www.jmlr.org/mloss/mloss-info.html}{\en{jmlr/mloss}}) κλπ.

\section{Αντικείμενο της διπλωματικής}
Ένας πολύ δημοφιλής τρόπος σύγκρισης γράφων, που αρχίζει να επικρατεί τα τελευταία χρόνια στο χώρο της μηχανικής μάθησης είναι οι ``γραφοπυρήνες'' (\en{graph kernels}).
Αυτά τα μέτρα ομοιότητας έχουν αποκτήσει θετική φήμη στην βιβλιογραφία τόσο για την μαθηματική θεμελίωση τους, που τους αποδίδει εγγύηση υπολογιστικής σύγκλισης (\en{computational convergence guarantee}) σε ποικίλα προβλήματα μάθησης, όσο και για την ύπαρξη και παρούσα ανάπτυξη μίας πληθώρας τέτοιων μεθόδων που η υπολογιστική τους πολυπλοκότητα ανήκει στην πολυωνυμική κλάση \en{P}.
Παρόλο που έχουν μελετηθεί πολύ τα τελευταία χρόνια και ένα μεγάλο πλήθος αυτών των τεχνικών θεωρούνται σημαίνουσες και καθιερωμένες, δεν επιχειρήθηκε με συστηματικό τρόπο η συλλογή τους σε ένα ελεύθερο λογισμικό, αντικειμενοστραφούς δομής, με δυνατότητα συλλογικής επεξεργασίας χρήσης από όλη την επιστημονική κοινότητα που να περιλαμβάνει ένα πλήρες εγχειρίδιο χρήσης/κατανόησης των τεχνικών αυτών για το ευρύ επιστημονικό κοινό.
Είτε για να καλύψει μία ανάγκη, είτε για να δημιουργήσει μία επιθυμία το \textbf{\en{GraKeL}} σχεδιάστηκε στο πλαίσιο αυτής της διπλωματικής, προκειμένου να ικανοποιεί αυτήν την απαίτηση.
Σχετικές απόπειρες (βλ. \cite{sugiyama2017graphkernels}) είναι ελλιπείς τόσο ως προς το περιεχόμενο και το εύρος απεύθυνσης τους αλλά και λόγω του ίδιου του τρόπου που είναι συσκευασμένος (\en{packaging}) ο κώδικάς τους.
Στο πλαίσιο ανάπτυξης του παραπάνω λογισμικού έλαβε χώρα η εκτενής μελέτη της υπάρχουσας βιβλιογραφίας των πυρήνων γράφων και η επιλογή των πιο σημαντικών, ενώ δόθηκε έμφαση στην βελτιστοποίηση της υλοποίησης τους σε επίπεδο κώδικα, τόσο όσον αφορά την αλγοριθμική τους πολυπλοκότητα, όσο και την πολυπλοκότητα \textit{υλικού} κατά την χρήση υπάρχοντων υπολογιστικών εργαλείων.
Παράλληλα με σκοπό να απευθύνεται σε ένα ευρύ κοινό προγραμματιστών, επιχειρήθηκε η συγγραφή ενός συστηματικού εγχειριδίου ανάγνωσης (\en{documentation}) τόσο για την χρήση όσο και για την συμμετοχή στην ανάπτυξη της ίδιας της βιβλιοθήκης.
Η γλώσσα προγραμματισμού που επιλέχθηκε ήταν η \en{python}, μία γλώσσα δημοφιλής τόσο σε επιστημονικούς όσο και σε τεχνολογικούς κύκλους.
Παρόλο που είναι χαρακτηρισμένη ως γλώσσα που ακολουθεί το πρότυπο του αντικειμενοστραφούς προγραμματισμού (\en{OOP}) από τους κατασκευαστές της, είναι γνωστή ως \textit{γλώσσα σεναρίων} (\en{script language}) από την προγραμματιστική κοινότητα μιάς και ακολουθεί εκτέλεση με διερμηνέα και οκνηρό (\en{lazy}) σύστημα τύπων, που δίνουν την ευκολία στον προγραμματιστή να αναπτύσσει γρήγορα \en{ad-hoc} εφαρμογές, ευδιάκριτα γραμμένες σε υψηλό επίπεδο (\en{high level}).
Παράλληλα το οικοσύστημα της \en{python} (\en{python ecosystem}), το σύνολο δηλαδή των βιβλιοθηκών που περιλαμβάνει η γλώσσα, τόσο από τους ίδιους τους κατασκευαστές τhς όσο και μέσω τρίτων στο επίσημο αποθετήριο βιβλιοθηκών γνωστό ως \en{PyPi}, επιτρέπει και προκρίνει τη συνύπαρξη τόσο γενικού όσο και ειδικού σκοπού βιβλιοθηκών, κάνοντας ταυτόχρονα πολύ εύκολη την εγκατάσταση τους.
Ταυτόχρονα πολλές δημοφιλείς βιβλιοθήκες επιστημονικού υπολογισμού (\en{scientific computing} - βλ. \en{Numpy}, \en{Scipy} κλπ) είναι είτε υλοποιημένες είτε εκτελέσιμες σε επίπεδο διεπαφής (\selectlanguage{english}interface\selectlanguage{greek}) μέσω συρραφής κώδικα (\selectlanguage{english}code wrapping\selectlanguage{greek}) άλλων γλωσσών, σε περιβάλλον προγραμματισμού \en{python}.

\section{Οργάνωση της διπλωματικής}
Η παρούσα διπλωματική εργασία χωρίζεται σε τέσσερα κεφάλαια. Μία θεωρητική εισαγωγή στο χώρο των πυρήνων γράφων, καθώς και η παρουσίαση όλων των επιλεγμένων πυρήνων δίνεται στο κεφάλαιο \ref{chap2}. Η ανάλυση της σχεδίασης του λογισμικού, καθώς και της συσκευασίας και διανομής του γίνεται στο κεφάλαιο \ref{chap3}, ενώ η πειραματική του αξιολόγηση παρουσιάζεται στο κεφάλαιο \ref{chap4}. Τέλος στο κεφάλαιο \ref{chap5} παρουσιάζεται η πειραματική αξιολόγηση του λογισμικού καθώς και τα συμπεράσματα της διπλωματικής, ενώ παρέχονται ιδέες και κατευθύνσεις για την προοπτική μελλοντικής εξέλιξής του.
