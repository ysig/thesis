\begin{abstract}
Το πρόβλημα ακριβούς μέτρησης της ομοιότητας μεταξύ δεδομένων που έχουν αναπαρασταθεί με τη μορφή γράφων είναι στον πυρήνα πολλών εφαρμογών σε ένα μεγάλο εύρος επιστημονικών και τεχνολογικών κλάδων.
Λόγω της πολυωνυμικής υπολογιστικής πολυπλοκότητας και της θεμελιώδους θεωρητικής τους βάσης, οι πυρήνες γράφων έχουν εμφανιστεί ως μία ελπιδοφόρα προσσέγγιση στην αντιμετώπιση αυτού του προβλήματος.
Εστιάζοντας σε διαφορετικά δομικά χαρακτηρηστικά των γράφων, μπορούν στην πολυμορφία τους να παρέχουν μία λύση αιχμής σε πλήθος εφαρμογών του πραγματικού κόσμου.
Σε αυτήν την διπλωματική παρουσιάζουμε την ανάπτυξη του \en{GraKeL}, μίας βιβλιοθήκης που ενοποιεί μία ικανή ποσότητα σημαντικών πυρήνων γράφων σε μία κοινή αντικειμενοστραφή δομή.
Η βιβλιοθήκη είναι υλοποιημένη σε γλώσσα προγραμματισμού \en{Python} και είναι κατασκευασμένη βάσει του προτύπου της βιβλιοθήκης \en{scikit-learn}.
Είναι εύκολη στη χρήση και μπορεί να συνδυαστεί φυσικά με υπολογιστικά αντικείμενα του ίδιου του \en{scikit-learn} για να σχηματίσει μία πλήρη ακολουθία εφαρμογών μηχανικής μάθησης, για προβλήματα όπως αυτά της ταξινόμησης και της συσταδοποίησης γράφων.\\
Παρέχεται με άδεια \en{BSD} και μπορεί να βρεθεί στη διεύθυνση: \en{\url{https://github.com/ysig/GraKeL}}.
\begin{keywords}Ομοιότητα Γράφων, Πυρήνες Γράφων, Βιβλιοθήκη \tl{Python}, Βιοπληροφορική, Χημιοπληροφορική
\end{keywords}
\end{abstract}


\begin{abstracteng}
\tl{The problem of accurately measuring the similarity between graphs is at the core of many applications in a variety of disciplines.
Because of their polynomial complexity and their fundamental theoretical foundation, graph kernels have recently emerged as a promising approach to this problem.
By focusing on different structural aspects of graphs, in their diversity they can provide state of the art solutions to real world applications.
In this thesis, we present the development of GraKeL, a library that unifies a sufficient amount of influential graph kernels into a common object-oriented framework.
The library is written in Python programming language and is build on top of the scikit-learn project template.
It is simple to use and can be naturally combined with scikit-learn's modules to build a complete machine learning pipeline for tasks such as graph classification and clustering.\\
It is BSD licensed and can be found at: \url{https://github.com/ysig/GraKeL}.}

\begin{keywordseng}
    \tl{Graph Similarity, Graph Kernels, Python Library, Bioinformatics, Chemoinformatics}
\end{keywordseng}
\end{abstracteng}

\begin{acknowledgements}
Θα ήθελα να ευχαριστήσω τον επιβλέποντα καθηγητή κ. Ανδρέα Σταφυλοπάτη για την υποστήριξη του.
Επίσης θα ήθελα να ευχαριστήσω τον καθηγητή Μιχάλη Βαζιργιάννη για την κοπιώδη υποστήριξη και το ενδιαφέρον όλο το διάστημα που εργαζόμουν μαζί του στο Εργαστήριο \en{LiX}, στο Παρίσι. 
Ακόμα ευχαριστώ ιδιαίτερα τον μεταδιδακτορικό φοιτητή Γιάννη Νικολέντζο δίχως τη βοήθεια, την καθοδήγηση και το ενδιαφέρον του οποίου δεν θα είχα φέρει εις πέρας το δύσκολο έργο ολοκλήρωσης αυτής της διπλωματικής.\\
Τέλος θα ήθελα να ευχαριστήσω τους γονείς μου Ελένη και Παναγιώτη για την αγάπη και την συνεχή παρουσιά τους και τις φίλες και τους φίλους μους δίχως την ύπαρξη των οποίων δεν θα υπήρχα.\\
Η παρούσα εργασία είναι αφιερωμένη στην μνήμη της γιαγιάς μου Μαρίνας και του ξαδέρφου μου Παντελή.
\end{acknowledgements}

